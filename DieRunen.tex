\documentclass[11pt,titlepage,a5paper]{book}
  \usepackage[T1]{fontenc}  
  \usepackage[ngerman]{babel}
  \usepackage[utf8]{inputenc}
  \usepackage{graphicx}
  \usepackage{geometry}
  \usepackage{pstricks}
  %Musiknoten 
  \usepackage{wasysym}
  \usepackage{amssymb}
  \usepackage{allrunes}
  % \usepackage[style=numeric-comp,backend=bibtex]{biblatex}
  % \addbibresource{Bilbliothek.bib}

\newcommand{\sterne}{\par{\centering ***\par}}
  %\newcommand{\ti}{"`}
  %\newcommand{\ho}{"'}
  % \newenvironment{tg}{\begin{quote}\em}{\end{quote}}
\newcommand{\am}{Amélie }
\newenvironment{dichter}{\begin{flushright}}{\end{flushright}}

\newcommand{\changefont}[3]{
\fontfamily{#1} \fontseries{#2} \fontshape{#3} \selectfont}
  
\pagestyle{plain}
\clubpenalty3000
\widowpenalty3000
\displaywidowpenalty=3000
  %\frenchspacing 
\parindent0em 
\setlength{\parskip}{0.4ex plus 0.6ex minus 0.4ex}
\sloppy
\geometry{left=2.5cm,right=2cm,bottom=3.5cm,top=2.0cm}



\author{von\\[0.8ex]Felicitas Stotz}
\title{\bf\Huge Von Bäumen und Menschen}



\begin{document}


\maketitle

\chapter{Vorwort}

\textarc{[\withlines]futark}

Die Rauhnächte sind für mich jedes Jahr eine aussergewöhnliche Zeit. Die Legenden, die sich um diesen Zeitraum ranken, müssen wahr sein. Mir sind viele wunderliche Dinge passiert in diesen magischen Tagen, Jahr für Jahr\dots

Die Rauhnächte 2019/2020 waren bisher die intensivsten. Die Bäume in meiner Umgebung begann mit mir zu reden. Mit Bäumen zusammen zu träumen, das Wetter anzuschauen, sie zu spüren, das alles war nichts neues für mich.

Jedoch mit einem Schreibheft und Stift bewaffnet an verschiedenen Bäumen zu lehnen und wie eine Verrückte aufzuschreiben, was ich in einem Gespräch mit ihnen erfuhr. Das war sehr neu. Sowohl die Klarheit in den Gesprächen, also auch die Möglichkeit diese aufzuschreiben. Vorher war mir das nicht möglich, weil ich mich auf den Kontakt zum Baum selbst konzentrieren musste. Nun war die Gesprächsleitung so stark, dass ich nebenbei schreiben konnte.

Die Bäume begannen mir Runenbilder zu schicken. Das war schwierig, denn ich kannte mich mit Runen nicht aus! Ich wusste, dass es sie gibt, da mich die nordischen Göttergeschichten seit Kindertagen faszinierten, aber die Runen interessierten mich nicht so sehr, da sie mir nur als Orakel bekannt waren.

Ich versuchte bisher mit möglichst wenig Material auszukommen und mich vor allem auf die Dinge zu konzentrieren, die ich immer bei mir habe: Meinen Körper. So war ich überrascht, dass mir die Pappel eine Rune schickte.

Ich hatte prompt ein falsches Bild aufgeschnappt und musste mir erstmals die Bilder der Runen bekannt machen. Dann erfuhr ich sehr viele spannendes und hilfreiches Wissen über Runen und wie sie Menschen und Bäumen die Interaktionen, ein Gespräch erleichtern können.

Dieses Wissen gebe ich hier weiter auch auf Wunsch der Bäume hin, die darauf hoffen mit mehr Menschen, mit allen Menschen wieder, ja genau, wieder in Kontakt zu kommen, sich auszutauschen!

Es ist mir folgendes wichtig zu sagen: Ich verstehe mich in keiner weise als Medium und ich channel nicht!

Ich habe seit meiner Kindheit einen intensiven Bezug zur Natur und fühlte mich mit ihr immer vertraut und beschützt. Ich bewegte mich viel in der Natur und sass stundenlang an einem Ort und lauschte dem, was um mich herum war\dots Ich beobachtete das Wetter, die Wolken, die Farben des Himmels und den Geruch der Luft.

Ich blieb an dem Thema dran, beschäftigte mich mit allem, was mir an Büchern über Natur, Naturwesen sinnvoll erschien und probierte Übungen aus. So kam ich Schritt für Schritt voran. Mein Weg führte mich in die Ausbildung von Anouk Claes und Elizabeth Guo, die beide auf ihre Art, meine Möglichkeiten mit der Natur zu kommunizieren in wenigen Jahren potenzierten.

Ich werde meine Interaktionen mit den Bäumen Gespräche nennen. Ich habe das Glück, dass ich die Bäume tatsächlich nicht nur auf sehr vielen Ebenen wahrnehme, sondern mit ihnen, zugegeben mit viel Übung, wirk-lich reden kann. 

Die Bäume sprechen nicht physisch. Sie haben keinen Mund. Aber sie können auf verschiedenen Wegen kommunizieren. Ich kann ihnen Fragen stellen und sie stellen mir Fragen, es ist ein Gespräch, wie Menschen es manchmal führen, wenn sie miteinander von einem Thema begeistert sind, wenn sie von einander begeistert sind und sich intensiv austauschen.

Es wird, je nach Temperament des Baumes gescherzt, gegluckst, dazwischen gerufen, ernsthaft diskutiert, gelehrt, gelernt, mütterlich ermahnt und frech provoziert, empathisch gelauscht\dots

Jeder von uns hat die Möglichkeit mit der Natur, den Tieren und den Bäumen und Pflanzen diesen Austausch zu lernen und aus der Einsamkeit des rationalen Verstandes in eine alles umfassende Wirk-lichkeit einzutauchen.

Es ist der Wunsch der Bäume und mir eine Brücke für die zu bauen, die diese Wirklichkeit ahnen, spüren und besser erleben möchten und ihnen auf diesem Weg zu helfen.

Ich war ziemlich überwältige nach der Fülle an Baumgesprächen und erzählte einer Bekannten davon. Sie fragte sofort: ,,Oh! Mit den Bäumen geredet? Haben die nicht fürchterlich geschimpft über uns Menschen, weil wir soviel Dreck machen?'' 

Nein, die Bäume haben nicht geschimpft. Sie haben mich mit offenen Ästen (Hihi) empfangen und waren begeistert einmal wieder mit einem Mensch zu plaudern. Sie reichten mich herum, schliesslich konnten sie nicht zu mir kommen: Geh' noch zu dieser Weide, und zu jenem Hasel und die Kirsche möchte auch etwas los werden und die Birke weiss zu dem Thema\dots so ging es 9 Tage in den Rauhnächten\dots

\chapter{Einleitung}

Während ich diese Zeilen schreibe, stelle ich mir vor, dass Du noch nicht mit so vielen Bäumen gesprochen hast, aber den Wunsch dazu hast. Ich stelle mir vor, dass Du Lust hast mit mir auf die spannende Reise in das Universum der Bäume\footnote{ja so gross ist es!} zu kommen und mit ihnen und den Runen den Geheimnissen, Geschichten und Wundern aus Menschen- und Baumwelt wieder neues Leben einzuhauchen.

Wir Menschen empfinden uns seit langem getrennt von den Bäumen, der Natur, den Tieren. Wir sind es, die sich trennen! Und wir müssen nicht weit gehen, um die Trennung rückgängig zu machen. Wir haben alles, was wir brauchen, um mit der Natur, den Bäumen zu reden schon bei uns. 

Wir können jederzeit entscheiden, ob wir uns wieder verbinden wollen, oder ob wir wegschauen wollen. Aber, ja aber, ich sage nicht das es leicht wird. Es ist einfach, aber nicht leicht. Denn wir müssen dieses Gefühl der Verbundenheit wie einen Muskel trainieren und es geht uns dabei genau so unbeholfen, wackelig und anstrengend wie damals als wir Laufen lernten, Schwimmen oder Fahrradfahren.

Deshalb werde ich Dir, bevor wie zu den Runen kommen und wie wir mit ihnen mit den Bäumen reden können, einige Tipps und Tricks aus meinem Erfahrungsschatz an die Hand geben, um Dich zu unterstützen Deine Sprache, Deinen Weg zu den Bäumen zu finden.

Der erste Teil des Büchleins wird jeweils viele kleine Kapitel enthalten. ich gebe Dir meine Werkzeuge an die Hand und falls Du das eine oder andere selbst schon benutzt, kannst Du die Kapitel überspringen.

Die Kapitel über Bäume wir mein Wissen, dass ich im Laufe meiner Freundschaft mit ihnen sammelte enthalten. Es kann sein, dass es sich von Deinen Erfahrungen unterscheidet. Das ist gut und richtig, denn die Erde hat so eine grosse Vielfalt, dass sie ein einzelner Mensch nicht erfassen kann. Wenn Du eigene Erfahrungen hast, dann benutzte die meinen als Ergänzung.

Die Kapitel der Runen wird die Informationen enthalten, die ich aus den Gesprächen mit den Bäumen herausgefunden habe. Ich habe einige Bücher zu dem Thema bestellt und bemerkt, dass die Form, wie ich die Runen kennengelernt habe, sich z.T. sehr unterscheidet von dem, was in den Büchern steht.

 Auch hier gilt, wie in den Baumkapiteln, die Vielfalt mit der wir die Muster und Bewegungen der Runen benutzen können, ist riesig. Ich geben Dir eine mögliche Variante zur Hand. Du kannst sie ausprobieren, teilen und auch mit Deinem Wissen ergänzen und erweitern. Im geistigen Bereich ist es sehr nützlich nicht in entweder/oder zu kategorisieren, sondern sowohl/als auch zur Prämisse zu machen. 
 
Ich wünsche Dir viel Spass daran und Erfolg!

\chapter{Bäume und Menschen}

Der Mensch ist Natur. Und in der Natur besteht alles aus Interaktion. Jeder und jede ist mit jedem verbunden. Das ist so, weil es zum Leben und zum Sterben dazugehört. Diese Bereiche sind für alle die wichtigsten Themen: Schutz, Nahrung, Sex, Wetter, Gefahren, Krankheit und Tod.\footnote{Wenn Du es nicht glaubst, dann schau Dir die News, Twitter, Facebook, Tinder oder Youtube an.}

Das, was tatsächlich spannend daran ist, sind Zweierlei: Zum einen der Eindruck, den wir Menschen haben, dass wir durch unseren rationalen Verstand nicht mehr an diese Themen des Lebens gebunden  sind. Und als zweites, dass wir vergessen haben, dass ein Teil von uns immer in Interaktion mit der Natur ist.

An dieser Stelle  möchte ich Deine Aufmerksamkeit auf diese Verbundenheit und den Austausch mit der Natur lenken.


\chapter{Baum und Mensch}

Wir Menschen haben viele Werkzeuge mit denen es uns möglich ist mit unser Umwelt in Aktion zu treten. Diese Werkzeuge sind sehr unterschiedlich, weil wir auf sehr vielen Ebenen verbunden sind und uns austauschen. 

Unser Körper allein hat schon unzählige Werkzeuge zur Hand, aber schliesslich ist er dafür gemacht in der Umwelt, die viele Tausende Jahre aus dem bestand, was wir Natur nennen, zu überleben. Im Laufe der Zeit haben wir begonnen eine räumliche Trennung zu ziehen zu der Natur und unserem Bereich, wo wir uns hauptsächlich aufhalten. Wir machen einen Unterschied zwischen Natur- und Kulturraum.

Die Bäume treffen wir in beiden Bereich, sie sind Teil der unserer Umgebung und unserer Kultur. Die Bäume waren lange Zeit im Austausch mit den Menschen. sie sind bekannt mit den Geschichten, Märchen und Mythen und nutzen diese, wenn sie mit uns sprechen, um auf zusammenhänge aufmerksam zu machen.

\chapter{Der Baum}

\section{Baumträume}

Wenn es dem Baum gut geht in seiner Umgebung, dann träumt er. Das ist etwas, dass alle Pflanzen machen. Sie träumen sich in den Himmel und in die Erde. Sie lassen sich vom Wind die Blätter streicheln und wiegen ihre Äste. Sie bekommen über ihre Wurzeln und die Pilze die dort und im Boden ihr Netz ausgebreitet haben alles, was sie brauchen an Nährstoffen, an Informationen.

Es ist schwierig ein Wort dafür zu finden für die Dinge, die die Erde, die Pilze und die Wurzeln austauschen. Diese Informationen sind nicht so, wie wir Informationen sammeln, wenn wir z.B. im Internet googlen oder Wissen sammeln. 

Du wirst jetzt lachen, wenn ich sage, Bäume haben kein Gehirn. Aber das macht einen riesigen Unterschied aus, ob ich die Dinge, die um mich herum passieren über Wurzeln und Pilze verarbeite oder, ob ich das Wissen über ein Gehirn serviert bekomme!

Wenn ich kein Gehirn habe, dann habe ich keinen rationalen Verstand, der in Worten denkt, sondern ich nehme mein Wissen direkt aus dem geistigen Raum.

Geistiger Raum? Mmh, Stelle Dir vor: Alles, alles, also wirk-lich alles hat eine unsichtbare Entsprechung. Du hast nicht nur Deinen Körper den Du anfassen kannst, sondern genau an der gleichen Stelle, wo z.B. Deine physische Hand ist, hast Du auch eine feinstoffliche Hand, die du nur nicht sehen kannst. Denn spüren kannst Du sie! 

Übung:

 Nehme einen Gegenstand, z.B. ein Glas. Zu Beginn ist es am besten einen einfachen Gegenstand zu nehmen, der aus einem Material besteht. Material ist egal, Plastik geht z.B. auch. Du nimmst diesen Gegenstand in die Hand und stellst ihn wieder hin. 

Nun nimmst Du ihn ein zweites mal in die Hand und stellst Dir aber dabei vor, dass sich Deine feinstoffliche, oder eben geistige Hand zusammen mit Deiner Hand um den Gegenstand schliesst. 

Wenn Du den Eindruck hast, dass du mehr spürst als beim ersten mal, dann bist Du auf einem guten Weg. 

Wenn Du Dir bei einem dritten mal, wo Du den Gegenstand  greifst, vorstellst, wie sich Deine feinstoffliche Hand erst öffnet und dann am besten langsam um den Gegenstand schliesst, zusammen mit Deiner Physischen Hand natürlich, dann wirst Du den Eindruck haben, Deine physische Hand sei wie angeklebt, angesaugt an den Gegenstand.

Du kannst dieses Phänomen gut bei kleinen Kindern beobachten, die alle Dinge so  anfassen und dementsprechend manchmal nicht loslassen. Oder auch bei Sportlern, die verrückte Dinge mit dem Ball anstellen, wobei Du den Eindruck bekommst, der Ball klebt an der Person oder wird wie angezogen.

Also, das ist der geistige Raum. Und dort sind die Dinge verbunden. Auch unsere Gedanken sind unsichtbar. Ja! Diese gehören in den geistigen Raum mit hinein. Allerdings gibt es verschiedene Quellen von Gedanken (Siehe Anhang)

\section{Wie rede ich mit einem Baum?}

Am Besten suchst Du Dir einen Baum in Deiner Nähe aus. Einen den Du schon oft angeschaut hast, oder in dessen Schatten Du schon gesessen bist. Dieser Baum kennt Dich dann und ihr könnt leichter in eine Interaktionen kommen.

Zu Beginn ist es gut, Zeit mitzubringen 20 Minuten und vielleicht einen Baum zu wählen, wo Du Dich nicht so beobachtet fühlst von anderen Menschen in der Zeit des Gespräches.

Es ist eh für Deinen rationalen Verstand, Dein Ego\footnote{Siehe Anhang} schwer einzusehen, warum Du mit einem Baum reden willst. Und wenn Du Dir zusätzlich Gedanken machst, was wohl die anderen Menschen, die vorüber gehen, denken könnten, ist das nicht sehr hilfreich. Rechne damit, dass ein Teil von Dir das alles höchst peinlich oder verrückt findet. Das ist gut! Ignoriere es, versuche nicht den Teil von etwas zu überzeugen\dots

Du willst mit dem Baum reden nicht mit Deinem Gehirn!

Suche Dir eine bequeme Position, Sitzplatz, wo Du die nächsten 20 Minuten gut sein kannst. Du kannst dicht am Baum sein und ihn berühren. Du kannst im Bereich seiner Wurzeln sein, oder in Sichtkontakt.

Was braucht es noch? Eine Telefonleitung!

Die baust Du Dir folgendermassen: Du konzentrierst Dich auf Deine Herzgegend. Du lenkst Deine Aufmerksamkeit dorthin. Die Stelle wirst Du deutlicher spüren, sie wird sich wärmer anfühlen. Dann legst Du Stück für Stück eine Leitung, ein Kabel durch Dich hindurch zu der Hand, dem Fuss durch den Du telefonieren willst durch die Erde/ durch die Hand in die Baumrinde zum Baum\footnote{Siehe Abbildung}.

Es ist eine richtige Leitung, d.h. es ist wichtig, dass sie von Deiner Herzgegend durch Dich hindurch geht und durch den Boden, oder in einer direkten Verbindung Hand/Baum. Wenn Du die Leitung unterbrichst, oder sie nicht durchgängig zum Baum führt, dann kannst Du nicht telefonieren.\footnote{Es gibt verschiedene Methoden, vielleicht kennst Du eine andere. Ich empfehle diese, weil sie direkt ist und dadurch intensiv.}

Probiere es aus und lasse Dich nicht entmutigen, wenn es am Anfang schwierig ist sich diese Leitung in Deiner Aufmerksamkeit zu bauen und zu halten. 20 Minuten können sich am Anfang lang anfühlen. Dann mache es erst kürzer 5 Minuten, 10 Minuten\dots Du wirst spüren, dass das Zeitempfinden sich eh verändert. 

20 Minuten sind eine gute Zeitspanne, damit Du nicht nur intensiv mit dem Baum interagieren kannst, sondern auch, um Deine Aufmerksamkeit selbst besser kennen zu lernen. Diese wird nicht durchgehend gleich bleiben, sondern sich in Wellen bewegen. Es ist gut, wenn Du Deine innere Bewegung kennst, wenn Du mit Dingen telefonierst.

Du kannst mit dieser Methode alles spüren, Dich mit allem austauschen. \footnote{Genau! Alles: Menschen, Tiere, Pflanzen, Steine, Gegenstände. Du wirst überrscht sein, was Dir z.B. eine Plastikflasche erzählen kann.}


\section{Was werde ich spüren, wenn ich mit dem Baum rede?}

Was passiert nun, wenn wir beginnen mit einem Baum in Interaktion zu gehen? Das kann ich Dir nicht sagen!

Das ist für mich geheim, denn das kannst nur Du für Dich herausfinden!

Dies ist eine sehr wichtige Sache:

Jeder Mensch ist ein einzigartiges Universum! Jeder Mensch wird daher etwas anderes spüren, etwas anderes mit dem Baum bei dem Gespräch bereden!

Deshalb beachte folgendes:

* Du kannst, wenn Du mit dem Baum verbunden bist, nichts falsch machen! Was Du bei Deiner Bauminteraktion spürst ist einzigartig!

* Du kannst es nicht wiederholen in dem Sinne, dass Du jedes mal das Gleiche spürst. Es ist wie im menschlichen Gespräch auch, wenn Du mit XY telefonierst, wirst Du vielleicht ähnliche Themen haben, aber die Worte, die Stimmung, etc. ist jedes mal anders.

* Alles hat seine eigene Schwingung, Bewegung, Muster und diese wirst Du nach einer Zeit des Forschens spüren können. Die verschiedenen Baumarten, ja jeder Baum, auch von der gleichen Sorte, ist einzigartig. Und dennoch haben z.B. alle Birken einen Teil der ähnlich ist, der die Birken, als Birken ausmacht.

* Ein anderer Mensch wird, wenn er mit ,,Deinem'' Baum redet evetuell etwas ähnliches spüren wie Du, denn ihr telefoniert ja mit dem selben Baum, aber\dots

es kann auch etwas völlig anderes sein! Wichtig! Ihr habt beide alles richtig gemacht! Jeder Mensch ist ein einzigartiges Universum, mit einer einzigartigen Schwingung und so\dots

können wir uns im geistigen Bereich ergänzen!
Das ist wichtig! Du wirst keinen finden, der den Baum so spürt wie Du! Aber, Du machst alles richtig und der andere auch!

Schön ist es, wenn Du Menschen findest mit denen Du Dich austauschen kannst. Denn wenn Dir Deine Freundin/Freund erzählt von ihrem Baumgespräch, dann kannst Du ihre Sicht spüren und wirst durch sie bereichert.

Wir Menschen haben oft den Eindruck, wir müssten alle das gleiche spüren, damit es richtig ist. Das ist nicht so. Jeder wird die Welt ein Stück anderes erleben.

Stell Dir vor, die Welt, ein einzelner Baum, ein kleines Sandkorn, sie alle können auf so verschiedene Arten angeschaut werden wie es Menschen auf der Erde gibt! Und dass ist gut, denn erst, wenn wir alle unsere Interaktionen austauschen, dann bekommen wir ein ganzes Bild. 

Stell Dir vor alles/jedes ist ein Puzzle und jeder Mensch hat 1 Puzzlestück! Wir können also z.B. die Ganzheit eines Baumes erst erfahren, wenn alle Menschen ihr einzigartiges Puzzleteil beisteuern. Deshalb ist es sehr spannend, bereichernd und lustig die Baumgespräche mit mehreren Menschen zu machen.

Aber\dots

\section{Kann ich mich vorbereiten damit ich den Baum gut spüre?}

Es gibt verschiedene Möglichkeiten, die Dir und dem Baum helfen können, dass ihr einander gut versteht.

1) Und das Wichtigste: Üben, üben, üben! Und nicht nur mit dem Baum, sondern mit möglichst vielen unterschiedlichen Dingen, Lebewesen, etc.

Denn nur so, wirst Du mit der Zeit herausfinden, was/wo/wie Deine Telefonleitung und Du selbst funktionierst. 

Jeder ist einzigartig, deshalb gibt es verschiedene Leitungen: Die eine lässt Geräusche durch, die andere Bilder, wiederum eine geometrische Muster, eine weitere Ordnung, eine Strukturen, bei der einen entstehen im Körper Gefühle, oder ein Vibrieren, Kribbeln, \dots

Also: Üben, üben, üben! Damit Du herausfindest wie Du telefonierst.
Im Anhang sind Tipps wie Du Dein Körpergefühl besser kennen lernen kannst.

2) Beschäftige Dich mit Deinem Gesprächspartner! Wenn Du mit Bäumen telefonieren willst, dann beschäftige Dich auf vielfältige Weise mit ihnen! Wichtig: Gehe in Deiner Beschäftigung dem nach, was Dich im Zusammenhang mit Bäumen berührt!

* Bücher\footnote{Ich bin Bücherwurm, deshalb laange Bücherliste hinten, siehe Du-weisst-schon-wo!}, Wikipedia, Bilderbände, Filme, Naturgeräusche vom Wald, etc.

* Mache Dir die Geschichte von Bäumen und Menschen bewusst:
Du kannst einfach mal auf das Sofa liegen und innerlich eine Zeitreise machen und schauen, welche Bilder sich Dir zeigen: z.B. riesige Wälder, Robin Hood, Druidenhaine, unberührte Urwälder, Bauwerke, Künstler, etc. Menschen und Bäume leben Ewigkeiten zusammen. Wir haben eine gemeinsame Geschichte.

* Märchen, Mythen, Göttersagen. Können für Dich eine Fundgrube sein.

* Du bist mehr Bewegungsmensch, dann gehe joggen unter Bäumen, aber bewusst, schaue sie an. Gehe spazieren unter Bäumen.

* Mache Feuer, verbrenne Holz.

* Schnitze, werke etwas aus Holz.

* Mache Musik auf einem Instrument aus Holz.

* Berühre bewusst Deine Möbel aus Holz\dots

Auf diese Weise schaffst Du an Gesprächsstoff für Dich und Deinen Baum. Du gibst, aber vor allem Deinem Baum Futter, worüber ihr euch austauschen könnt! 

3) Vertraue Deinem Gesprächspartnerbaum! Er wird sich freuen mit Dir zu telefonieren! Und im Gegensatz zu Dir, hast Du es mit einem Telefonprofi auf geistiger Ebene zu tun! Alle Pflanzen und Tiere, Steine sind Telefonprofis! Denn sie telefonieren die ganze Zeit. (Wir Menschen auch, nur wir haben uns angewöhnt den, meistens langweiligen, sich im Kreis drehenden Geplapper unseres rationalen Verstandes zuzuhören.)

Was bedeutet das für Dich? Du kannst sicher sein, Dein Baum wird sich alle Mühe geben, damit Du ihn gut verstehen kannst! Er spürt Dein Muster, Deine Schwingung, aber spürt auch die Bilder, Gedanken, die Du Dir über Holz, Bäume machst und er wird z.B. die Informationen, die er bei Dir spürt, die Du in 2) gesammelt hast, bemerken und Dir optimal darauf antworten! Immer!

Das bedeutet für Dich, wenn Du den Eindruck hast, Du bemerkst nichts, dann glaube nicht, es geht nicht, blabliblu\dots Bopp! Durchatmen, lächeln, und\dots

Entspanne Dich und probiere es einfach zu einem anderen Zeitpunkt! Es braucht eine Zeit. Aber, Du hast in Deinem Baum einen Vollprofi an Deiner Seite! Ihr werdet es gut schaffen!

Du wirst jetzt vielleicht denken, hei, sollte es nicht um Runen gehen? Richtig!

\chapter{Von Runen, Bäume und Menschen}

\section{Runenvater Odin}



Veit ek, at ek hekk\\
vindga meiði á\\
nætr allar níu,\\
geiri undaðr\\
ok gefinn Óðni,\\
sjalfr sjalfum mér,\\
á þeim meiði,\\
er manngi veit\\
hvers af rótum renn.\\


Ich weiß, dass ich hing\\
An windigem Baum\\
neun ganze Nächte,\\
vom Speer verwundet\\
und Odin geweiht,\\
ich selbst mir selbst,\\
an diesem Baum,\\
von dem niemand weiß\\
aus welcher Wurzel er sprießt.\\

So spricht Odin sind die Runen entstanden.

Ich weiss nicht, was der Historiker, der Wissenschaftler dazu sagt oder herausgefunden hat. Ich versuche die Quelle einer Sache zu finden, oder zuschauen, ob das involvierte Wesen selbst etwas dazu gesagt hat. 

Odin hat etwas gesagt und ich finde, es gibt wenig daran zu deuteln. Es tönt so, wie es tönt, wenn der Baum mit einem spricht, wenn er seine Blätter im Wind erzählen lässt, wenn sich der Gott, der Baum, der Mensch in tiefer Interaktionen miteinander befinden\dots Die vielfältigen Bedeutungen des Textes können Dir einen Einblick geben, wie Mensch und Baum, Baum und Gott, Wesen zu Wesen kommunizieren.

Was für usn wichtig ist, Odin brachte aus diesem Einswerden mit dem Weltenbaum die Runen mit. Neun tage und Nächte brauchte es, bis der Baum Odin das Geheimnis der Runen raunen konnte, Neun Tage und neun Nächte sprechen die Bäume selbst im Schlaf mit mir, um mir mich zu lehren, wie ich mit Hilfe der Runen einfacher mit ihnen sprechen kann. Und sie sagten: ,,Schreibe auf, bringe es auch denjenigen bei, die es wissen möchten!''

Die Runen, so erzählte mir die Pappel, sind jedem Baum bekannt. (Ob sie alle Bäume, oder die im europäischen Kulturraum meinte, müssen wir noch herausfinden.)

Die Runen bestehen aus einfachen Mustern, die symbolisch eine Bewegung anzeigen. Allerdings sind diese Bewegungen in Beziehung zu setzen im Raum. Allerdings, da es sich um ein Symbol, ein Muster handelt, beinhaltet es je nach der Ebene auf der ich es anschaue, sehr viele Aspekte.

Auf diese Weise kann ich mit einem einfachen Bewegungsmuster sehr komplexe Themen, Bilder und Bewegungen übermitteln. Und genau das ist es, was die Bäume mir mit den Runen an die Hand geben möchten und allen, die Interesse daran haben mit ihnen Informationen auszutauschen.

Die Runen, so sagte die Pappelfrau, die Runen kannst DU Dir wie eine Art Telefon vorstellen, wenn Du mit einem Baum redest. Du kannst dem Baum eine Rune schicken und der Baum weiss dann mit welcher Nummer, bzw. mit welchem spezifischen Runenmuster, Runenthema Du mit ihm sprechen willst.

Du willst Dich mit dem Baum über Reichtum und alle Aspekte des Reichtums unterhalten? Dann schicke ihm Fehu, die Rune für die Bewegung, das Thema Reichtum \textarc{[\withlines]F} oder \textarc{F}. Es kann natürlich auch andersherum möglich sein und der Baum möchte mit Dir über diese Bewegung sprechen, dann wirst DU, während Du mit dem Baum in Interaktion bist, die Rune \textarc{F} in Deinem Körper wahrnehmen.

Je nachdem, wie Du am besten wahrnehmen kannst, kann das visuell, auditiv sein, aber auch eine Bewegung oder wir eine Berührung. Dies wirst Du im Laufe der Zeit bemerken, wie Du wahrnimmst \footnote{Siehe Anhang}

Jeder Baum ist einzigartig, ähnlich wie wir Menschen. Die Bäume haben in der Pflanzenwelt auch eine ähnliche Funktion wie wir Menschen für alle Lebewesen haben. Die Bäume sind in verschiedene Baum- und Pflanzenfamilien eingeteilt. Es kann, muss aber nicht, hilfreich sein, sich diese anzuschauen. Mit dieser Familie bekommt eine Gruppe von Bäumen eine gemeinsame Grundfärbung. 

Z.B. gibt es zwischen allen Weiden eine Gemeinsamkeit, wenn Du also mit einer Weide interagierst und mit Weiden schon vertrauter bist, dann wird Dir ein Teil der neuen Weide bekannt vorkommen und ein Teil wird die Einzigartigkeit genau dieser Weide ausmachen und etwas neues sein.

So ist es für uns Baumspürer gut mit verschiedenen Bäumen aus verschiedenen Familien zu sprechen und mit verschiedenen Bäumen aus derselben Familie. So kannst Du dann die Familie und die Einzigartigkeit einzelner Bäume unterscheiden lernen.

Die verschiedenen Familien der Bäume, aber auch der einzelne Baum können mit bestimmten Runen verbunden sein. Du kannst Dir es so vorstellen, dass Rune und Baum intensiv an dem gleichen Thema interessiert sind.

Sehr deutlich wird das bei den zwei Runen, die sogar nach Bäumen benannt sind: Berkano \textarc{[\withlines]B}, \textarc{B} , die nach der Birke benannt ist und Iwaz/Eiwaz \textarc{[\withlines]I}, \textarc{I} die von der Eibe ihren Namen bekam.

Wie Du Dir denken kannst, wissen Eibe und Birke sehr viel zu den Themen, den Bewegungen von \textarc{[\withlines]B} und \textarc{[\withlines]I} zu sagen, aber sie kennen alle Runen und werden sie alle anwenden, wenn es ihnen gut erscheint.

So wollen wir uns zusammen auf den Weg machen die einzelnen Bäume besuchen und lauschen, was sie über die Runen zu wispern haben.\footnote{Dabei werden wir die Reihenfolge der Runen beibehalten, in der die Bäume mir ihr Runensysthem erklärt haben. Diese Reihenfolge ist eine andere als die des bekannten Futhark. Das soll Dich nicht irritieren. Sobald wir alle Runen auf dem Tisch haben, können wir die Runenalphabetische Reihenfolge nutzen. Das macht Sinn, da die Plätze der Runen im Alphabet tatsächlich eine Bedeutung haben.}

\section{Am Weltenbaum: Von Ratten und blinden Augen}

An diesem Abend (27.12.2019) sind wir zu Freunden zu ihrem traditionellen Weihnachstmenu eingeladen. Wir, mein Mann und ich fahren mit dem Fahrrad. das ist nichts besonderes. aber nun ist es Winter, es ist kalt, windig. es hat die Tage geregnet. Die Birs, der kleine Fluss der Basel und Basel-Land trennt, hat viel Wasser. Da wir gleich an seiner Mündung, dem Birschöpfli wohnen, fahren wir auf der Baseler Seite die Birs hinauf\dots

Der Weg ist unbeleuchtet. Kurz vor der Brücke, die Birsfelden und Basel verbindet, steht eine mächtige Eibe am befestigten Wegrand. Sie hat weitausladende Äste, die über dem Weglein einen kleinen, dunklen Tunnel bilden. Es ist ein grüner Tunnel, ein dunkler, grüner Tunnel.

Die Eibe steht auf einem Vorsprung der Uferböschung, die von einer Mauer gehalten wird. An dieser Stelle muss der Fahrradfahrer gut aufpassen. Er duckt sich automatisch unter den tief hängenden Zweigen und gleich da hinter kommt eine steile Abfahrt unter die breite Strassenbrücke hindurch. 

Ich habe Mühe mit der Sicht. Aber die Ratte, die an der Mauer unter dem Eibentunnel hockt, die sehe ich gut! Eine Ratte! und die Ratte, selbst erschrocken, läuft davon. Doch wohin soll sie, das arme Ding, springen? Eingeklemmt zwischen dem her anrollenden Rad und der Mauer? Natürlich, sie springt Richtung Rad\dots

Zum Glück geht alles so schnell! Die Ratte verschwindet an der Uferböschung wild sprudelnden, angefüllten Flusses, bevor ich genug Zeit hatte, selbst in den nahen Fluss zu steuern.

Ratten zählen nicht zu meinen liebsten Lieblingstieren. Aber ich achte sie. Und ich weiss, dass sie sehr scheu sind am Birschöpfli, obwohl es riesige Kolonien gibt unter den Steinen, unter dem Beton der begradigten Uferböschung, lassen sie sich selten blicken! \footnote{Hihi, was nicht bedeutet, dass sie nicht hurefrech die Chipstüten auf den Decken der abendlichen Sommerpartygäste leermampfen, wenn diese in ihr Scheierweia vertieft sind!} 

Ratte, Ratte, Ratte! Rattert es in meinem Kopf. Ich hüpfte für einen Moment aus mir heraus: Schamanenratte, Odins Ratte, Eiwaz-Eiben-Ratte!

Schon auf der Hinfahrt bemerkte ich, wie die Sicht meines rechten Auges immer schlechter wurde. das Auge schmerzte, der Kopf. Auf dem Heimweg goss es in Bächen auf uns nieder. Und ich sah nichts! Im Schneckentempo kroch ich heim, mit dem Fahrrad. Stocksteif, die Muskeln zum bersten angespannt und innerlich zitternd fuhr ich im Schritttempo blind durch den Sturm und den Regen.

Zuhause angekommen heulte es! Es war eine schreckliche fahrt gewesen. das Auge, es schmerzte. der Schmerz bohrte sich in den Kopf\dots

er blieb auch am Morgen. Am Morgen tat ich, was ich in den Rauhnächten tue, ich schreibe auf, was wichtig ist\dots die Ratte! Sie war wichtig.

Also sprach ich mit der Ratte, also suchte ich nach Hinweisen, die die Ratte mir gegeben hatte, weiter im Internet und in Büchern\dots
Es gab so vieles, was die Ratte erzählte: Z.B. das 2020 in China das Jahr der Ratte sei! und sie lobte die chinesische Sternendeuterkunst, die ihr Talent erkannt hatte und sie die chinesischen Sternzeichen als erste anführen liess!

Weiter\dots es fehlte noch etwas,\dots der Moment, in dem die richtige Informationen eine Kaskade von Aha-Momenten regnen lässt fehlte noch\footnote{Gell, Achtung! Alle Informationen sind wichtig! Wenn ich nur auf die achte, die ich einsortieren kann, dann achte ich nur auf die, die mir schon bekannt sind! Alle Informationen sind wichtig! dann gibt es die Informationen, die mit Deiner Absicht zu tun haben und denen wirst Du zuerst folgen und schauen, ob sie Dich weiter bringen.}

Die Ratte führte mich zu Eiwaz \textarc{[\withlines]I} Und wie schon am Abend an dem dunklen Ufer der Birs, führte die Ratte mich wieder zur Eibe. 

Die 13. Rune, die Rune der Versenkung in die Untere und die Obere Welt, die Rune, die die Geheimnisse der Götter, dem der es hören kann, zuraunt, die eröffnete den Reigen.

\section{Von der Pappelfrau und Ansuz \textarc{[\withlines]a} }

Doch bevor ich mich auf zur Eibe machte, besuchte ich meine Freundin die Pappel. Sie ist alt, weise, wunderschön und erfahren. Und ich stimmte mich bei ihr ein.

Ich fragte die Pappel nach den Runen und auch sie schickte mir eine. Es stellte sich dann später heraus, dass es diese Rune, so wie ich sie in meinem Kopf sah, nicht gab! Was schlicht daran lag, dass ich von Runen keine Ahnung hatte! Nicht wusste. wie die einzelnen genau aussehen. Ich fand die Rune, die die Pappel mir geschickt hatte mit ihrer Hilfe natürlich doch.

Diese Erfahrung half mir, weiter zu ergründen, wie Bäume und Menschen miteinander Dinge/Themen austauschen können. Es braucht hohe Sensibilität und Flexibilität. Und den innigsten Wunsch, den Dingen auf den Grund zu gehen. Die Bilder, Inhalte überprüfen und schauen, ob sie einen Anker finden, ob sie Sinn finden. Und mit denen, die Sinn gefunden haben weiter zu arbeiten und die, die keinen Sinn gefunden haben, stehen zu lassen! Denn, oho, vielleicht ergeben sie irgendwann, irgendwo einen Sinn, wenn Zeit und Ort dafür stimmen!

Die Rune, die mir die Pappel schenkte, oder mir übergab und mich einiges darüber lehrt ist \textarc{[\withlines]a} oder \textarc{a}. Ansuz. Ansuz bedeutet Mund. Es ist der die Schöpfung lobpreisende Mund, der Mund, der sich vorher in die Obere Welt hinaufgeschwungen hatte und nun berichtet, \dots

Der singt und dichtet von den Geheimnissen der Götterwelt! Die Pappel kennt sich aus mit dieser Welt, denn mit ihren luftigen Blättern beginnt sie ihr Lied zu singen, sobald sich ein Lufthauch regt.

eine staunende, aufnehmende Bewegung in die Obere Welt, die beschert und die Rune Ansuz \textarc{[\withlines]a}. Und eine zweite Bewegung, die der ersten Folgt, die das aufgenommene weitergibt.

Wenn Du Dir die Baumkrone vorstellst. Und die Blätter, Ästchen und Äste von oben betrachtest und Dir dann vorstellst, dass jedes Blatt ein Wahrnehmungsorgan ist, das den Himmel über sich spürt, dann hast Du einen Begriff davon, weshalb die Bäume eine Rune haben, wie Ansuz \textarc{[\withlines]a}. Die Bäume können über ihre Kronen den Sternen- und Planetenlauf spüren. Und es ist eine der Aufgaben der  Pappel, die ein grosser Solitärbaum ist und sehr viel Raum zur einnimmt, das, was sie über ihre grosse Krone und ihre luftigen Blätter aufnimmt, weiterzugeben an die anderen Wesen.

Über ihre Wurzeln ist die Pappel mit vielen Bäumen direkt vernetzt. Über ihre Duftstoffe, kann sie sich, je nachdem mit ihren Kolleginnen am anderen Birsufer auf der Basler Seite austauschen.

Vielleicht staunst Du jetzt doch, was so ein dumm, stumm, herumstehender Baum alles können soll\dots Jedoch, wenn Du Dich überwinden kannst, Dich dem Baum auf der wesenhaften Ebene zu nähern, dann kannst Du sehr wunder-volle Dinge erleben. Berichte mir dann davon, denn Du wirst etwas anderes erleben.

Ich lernte eine neue Seite der Pappel kennen. Nicht nur, dass sie mir eine Rune schickte, sondern einen neuen Raum. jedes Wesen, jeder Menschen ist eine, nein, mehrere Welten gross.

Und da Bäume ausgesprochen freundliche Wesen sind, werden sie sich bemühen, sobald sie Dich spüren, wenn Du mit ihnen in Interaktion trittst, Dir so verständlich wie möglich zu begegnen. Wenn Du also mit einem Baum redest und nichts hörst, dann ist es sicher gut, wenn Du locker bleibst und in Dich hineinhorchst, was sich seit Du den Baum spürst bei  Dir verändert hat im Körper.

Damit wir besser miteinander die komplizierten Runendinge austauschen konnten, erlebte ich folgendes: Ich lehnte mit dem Rücken am baum und vertiefte mich in unser Gespräch, da hatte ich den Eindruck eine Tür in meinem Rücken öffnete sich und eine zarte, aschblonde Frau mit langem Haar in einem weissen, wehenden Kleid, würde ich mich in den Baum ziehen. Dieser war im Inneren riesig, wie ein Palast. Jedoch unterhielt ich mich mit der Pappelfrau an der Türe. Sie war etwa so gross wie ich.

Dieses Wesen der Pappel hatte ich vorher noch nie gespürt, obwohl ich schon viele Male, einige Jahre mit der Pappel interagiert habe. Auch nach der intensiven Einweisungszeit der Rauhnächte und des Runenkurses, zeigt sich die Pappelfrau nicht mehr so häufig. Ihre Gestalt und Form half Baum und mir, Mensch miteinander spannende Dinge zu erforschen.


\section{Meister Reiher und die Hugin/Munins }

Die Tiere sind in den letzten Rauhnächten (2018/2019) zu mir gekommen, um mit mir Gespräche zu führen. Es war eine sehr intensive und schöne Zeit, die ebenso aufgeschrieben ist.

Dieses jahr unterstützten die Tiere die Gespräche mit den Bäumen. Sie begleiteten mich zum Teil, oder riefen laut von den Bäumen, oft wie zur Bestätigung oder um die Wichtigkeit bestimmter Worte zu unterstreichen.

Sie wiesen mich auf Übungen hin, die mich unterstützten die Aufmerksamkeit zu halten und die starke Interaktion zu tragen. Allen voran die Hugin/Munins, so nenne ich die Krähen, die immer mindestens zu Zweien ständig an meiner Seite waren.

Sie sassen auf den umliegenden Bäumen und beobachteten mich. manchmal tappten sie über die Wiese um mich herum. Sie sind die Augen Odins heisst es. Und da der Göttervater auch der Vater der Runen ist, war es kein Wunder, dass seine Vögel schauten.

Die ersten drei Tage an denen ich mit der Eibe sprechen wollte, stand der Reiher dicht am Ufer immer an der gleichen Stelle. Auch die Reiher kenne ich schon länger. Ehrlich gesagt, weiss ich nicht, ob es mehrere sind.

Der Reiher beherrscht, ähnlich wie ein Kung Fu Meister die Kunst der völligen Gelöstheit in dre Aufmerksamkeit. Es gibt nicht umsonst für Menschen Übungen im Kung Fu, die nach dem Reiher benannt sind.

Der Reiher kann Dir gut helfen, tief in Dir zu versinken und gleichzeitig in diesem neu dazu gewonnenen tiefen Raum alles wahrzunehmen, was um Dich passiert. Der Reiher besitzt die Fähigkeit andere Vögel herbei zu rufen, wenn er den Eindruck hat, sie könnten Dir etwas besser erklären.

Der Reiher half mir ruhig zu werden im inneren Raum und viel Platz zu machen. Den brauchte ich dann für die Eibe, die ein sehr mächtiger Baum ist. Doch um zur Eibe zu gelangen, kam ich bei der Rose vorbei\dots

Am Tag, an dem ich an die grosse Weihe gelehnt die Sonne genoss, stand der Reiher nicht am Ufer, sondern hatte sich auf einer kleinen Steininsel mitten im Fluss niedergelassen. Von dort aus sprachen wir, bevor ich zur Eibe weiterging.

Die Birs hatte viel Wasser und um den Reiher herum sprudelte, sprang und gurgelte das Wasser wild und vergnügt. Der Reiher führte mich über meinen Körper in die Ruhe, die er in sich trug und die er braucht, damit er in dem wilden Strom zu einem Fisch kommt.

Er zeigte mir ein Zentrum, das in jedem Wesen ist. Dieses Zentrum beinhaltet die Quintessenz des Wesens. Ich kann z.B. auch Dinge in meinem kleinen Zeh spüren, etc aber es gibt dieses eine Bewusstseinszentrum im Körper und das ist im Herzbereich.

Es ist ein feinstoffliches Zentrum. Es ist wie der kleine Punkt im Planetenzeichen der Sonne. \footnote{Siehe Abbildung} Es ist auf das Universum gesehen ein winzigkleiner Punkt, der durch sich mit allem verbunden ist und dadurch gleichzeitig unendlich gross ist\dots

Dieses kleine Zentrum ist unser Taktgeber, unser einzigartiger Musterweber, der mit jedem Pulsschlag das aktuelle Muster, die aktuelle Bewegung in das System schwingt und dann von allem Antwort erhält.

Ja, ich habe nicht gesagt, dass es einfach ist! Denn und das ist wichtig zu verstehen, jedes Ding/Wesen hat seine einzigartige Bewegung anhand der wir es von allem, auch den Familienmitgliedern (Z.B: Bei Bäumen, Menschen) unterscheiden können. Aber gleichzeitig passt sich interagiert dieses Muster ständig mit allem und jedem und ein Teil unserer Bewegung schwingt mit den Dingen mit, oder besser, unsere Spezifische Bewegung macht sich mit den einzigartigen Bewegungen aller anderen Dinge in der Umgebung bekannt und dadurch entstehen neue Resonanzbewegungen.

es ist wie die Wellen in einer Pfütze, wenn Du einen Stein hineinwirfst, erst gibt es Ringe, das wäre die spezifische Bewegung, die mit dem Äusseren kommuniziert. Aber dann trifft die Ringwelle auf das Ufer und rollt wieder zurück, während gleichzeitig Ringe hinauswandern, dadurch gibt es ein Interferenzmuster in den Wellen. Die Bewegungen des Ringes und die des Ufers bleiben gleich und ändern sich, gleichzeitig.\footnote{Siehe Abbildung}

Der Reiher stand also da im Fluss\dots

Er wies auf die Zeit hin, denn er stand dort, damit ich zur Eibe gehen konnte\dots




 
\section{Die schöne Rose und Fehu \textarc{[\withlines]F} }

Mit ihr hatte es angefangen. Die Rose, die frisch beschnitten worden war und ein winziger Strauch aus Dornen und wunderschön, zinnoberrot leuchtenden Hagebutten, hatte mich schon am ersten Weihnachtstag angequatscht als ich mit meinem Mann spazieren ging. Ich registrierte es, hatte aber nicht soviel Zeit, weil ich nicht alleine unterwegs war.

Ich versprach ihr wieder zu kommen. 

Wenn es Dir einmal so geht, dass eine Pflanze  oder ein Tier Dir ins Auge fällt, oder Du plötzlich irgendein Gefühl bekommst, das Du nicht einordnen kannst und Du hast nicht viel Zeit, dann komme in den folgenden Tagen nach Möglichkeit wieder. Die Pflanze kann Dir nicht folgen und so wirst Du sonst vielleicht ein nettes Gespräch verpassen! 

Die Rose gab mir ein selbstbewusstes, selbstbestimmtes Gefühl von sich. ,,Ich bin schön!'' so sagte sie: ,,Und ich zeige meine Schönheit, weil ich es bin und nicht um irgendetwas damit zu wollen oder jemanden/irgendetwas damit zu beeinflussen.'' 

,,Es ist!'' Und sie, die Rose ist bereit ihre Schönheit, ihre Früchte, die sie den Sommer über genährt hat und aus sich wachsen liess, zu !

Sei sagte: ,,Verschwende Deine Früchte nicht!

Gib sie nicht jedem, sondern gebe sie denen, die bereit sind, sich darum zu bemühen, die bereit sind sich für Deine Früchte an Deinen Dornen vorbei zu kämpfen, die bereit sind etwas dafür zu tun\dots , die bereit sind, sich von Dir berühren zu lassen!''

und die Dornen? Ich schaute zuhause im Wiki nach, die waren keine Dornen, sonder Stacheln. Ist das wichtig? Ja! Deshalb, wenn in Dir Fragen kommen, dann forsche nach Antworten, wenn möglich auf vielen Ebenen. Doch lassen wir die Rose selbst antworten:

,,Schau, wenn ich jemanden steche, dann fallen meine Stacheln einfach von mir ab! Ohne eine Wunde an mir zu hinterlassen! Meine Stacheln sind keine Dornen, die dann tief in meiner eigenen Haut stecken würden! Nein! Meine Stacheln, sie stechen und ich lasse sie einfach los\dots !'' Die Rose konnte mir im ersten Gespräch keine Rune sagen\dots

Die Rose zeigte mir eine Weibliche Gestalt. Sie hatte braune , lockige, lange Haare und ein rotes Kleid an. Sie wirkte viel physischer als die Pappelfrau, die fast durchscheinen war.

Ich kam ein weiteres mal zur Rose. Sie stand im Schatten und ihre Zweige und ihre Hagebutten waren  mit Raureif überzogen. ,,Schmücke Dich mit allem, was Dir zur Verfügung steht!'' sagte sie. ,,Überlege nicht, ob der Frost, der nun mal da ist, Deine Hagebutten kaputt macht, denn genau jetzt schmückt er sie mit Kristallen. Diamonds are the Girls bests Friends! Und wenn die Haut der Hagebutte kaputt geht, dann werden die Samen darin bearbeitet und transformiert.

Zuhause angekommen forschte ich weiter an den Runen. es ist sehr schwierig, denn es lassen sich viele, z.T. sehr schöne Internetseiten finden, Bücher, etc. Doch jede Informationsquelle weicht von der anderen ab. Was mache ich dann?

Ich vertraue darauf, dass ich die richtige finde und versuche eine gemeinsame Qualität heraus zu spüren, ein Muster zu spüren. Bei den Runen ist es sehr unterschiedlich, so dass ich es Dir überlassen muss, Deine eigene Seite, Dein eigenes Buch, etc. zu finden, wenn Du mehr über Runen wissen möchtest.

Die Menschen, die damals die Runen benutzten, hielten nichts vom Schreiben. Das könnte uns eventuell einen Hinweis geben, dass Runen auch einen anderen Zweck haben!

Sie sind plötzlich alle da. Aber vielleicht waren sie auch schon länger vorher da, nur sie wurden nicht zum Schreiben benutzt? das und dsa ist wichtig, wissen wir nicht!

Eine mögliche Möglichkeit ist, dass die Runen erst als Schriftzeichen verwendet wurden, als ihre Kraft und Bedeutung langsam in der nordischen Kultur verloren ging und die Menschen nicht mehr so genau wussten, wie sie verwendet werden.

Zu der Zeit, zu der die ersten Schriften mit Runen auftauchen, ca. 200 n. Chr. gab es im Mittelmeerraum schon andere Schriften, so dass die nordischen Menschen, die viel herum kamen und die anderen Schriften kennenlernten, vielleicht begannen mit ,,ihren'' Buchstaben, bzw. Runen auch zu schreiben. Vielleicht ahmten sie das Schreiben mit ihren Runen nach, nachdem sie mehr und mehr mit Schrift in Kontakt kamen. Und vergassen, wie die Runen eingesetzt worden waren, da die geheimen Praktiken längst verblasst waren oder nur von Eingeweihten praktiziert worden waren.

\begin{LARGE} Okay, okay, das ist 
voll Spekulation! Schreibe es Dir hinter die Ohren: Speculation!
\end{LARGE}

I love speculation! It is important! Because it's possible to find solutions in this way! Remember, cultivate! 

\begin{LARGE}
And never, never, never forget: Mister Dunning \& Mister Kruger!
\end{LARGE}

Runen sind ein Geheimnis, deshalb sei achtsam mit ihnen!

Also, zuhause angekommen forschte ich weiter an den Runen\dots

Ich lernte die Rune Fehu \textarc{[\withlines]F} kennen. 

\section{Der Odinsbaum Eibe und Eiwaz \textarc{[\withlines]I} }

Die Eibe, respektvoll näherte ich mich dem Baum Odins und machte mich bekannt. sie ist sehr kraftvoll. Sogleich wärmt sie mir den Rücken, den ich ihr zukehrte.

Wenn Du an einem Ort mit dem Bäumen sprechen willst an dem sich auch andere Menschen bewegen, dann machst Du es Dir selbst einfacher, wenn Du neben oder am Baum eine Position wählst, die Dich nicht die ganze Zeit denken lässt: ,,jetzt halten mich alle für verrückt!'' das ist wichtig! Du möchtest nicht, dass andere Dich für verrückt halten und Dich wohl möglich  meiden. Deshalb mache dem Teil von Dir, der in der Menschengruppe integriert sein will,die Freude und versuche eine Position zu wählen, die Dein Baumgespräch nicht zu offensichtlich macht. Ausser Du hast schon Übung und kannst Dich trotz der Blicke vertiefen.

Ich habe mich bei den Gesprächen mit der Eibe mit dem Rücken zu ihr gegen die kleine Mauer gelehnt. So stand ich zwar vielen etwas im Weg, aber sah aus, als würde ich die Sonne zwischen den Eibenzweigen geniessen.

Sobald ich mit mich auf die Eibe eingelassen hatte, entstand eine starke, senkrechte Bewegung in mir, die meine körperlich Begrenzung (Füsse/Kopf) völlig ignorierte.

Sie hatte ein fast grelles Licht, Leuchten. und die Eibe sagte mir, ihre starkes Gift entstehe durch ,,zu viel Lebendigkeit''. Sie ist zu lebendig für den materiellen Teil des Menschen. Ihr Strom, ihre Energie geht sehr weit nach oben und nach unten, was die Schamanen ja für ihre Reisen nutzen.

,,Wie oben, so unten!'' raunt die Eibe. Für sie gibt es keine obere und untere Welt, denn sie ist unsterblich, unbegrenzt => sie reicht ins Unendliche und kommt am anderen Ende wieder hervor. Dadurch entsteht ein unendlicher Kreislauf.\footnote{Siehe Abbildung}

Das Wasser, als Element des Gefühl, hat für sie kaum Bedeutung. Wenn ich mit den Bäumen und anderen Pflanzen zu sprechen beginne, dann frage ich sie gerne zu Beginn mit welchem der Elemente sie stark oder weniger stark verbunden sind. Natürlich sind Pflanzen immer mit allen Elementen im Kontakt. Aber wie sie mit den vier Elementen umgehen, das hat schon viel mit ihrem eigenen Charakter zu tun. Viele Forscher, Ärzte und Gelehrte der alten Zeiten beschäftigten sich schon intensiv mit dieser Frage.\footnote{Z.B. der ehrenwerte Meister Paracelsus, aber auch die Griechen, oder in neueren Zeiten C.G. Jung}

 Ihr Gift, das auf die Nerven wirkt, wirkt dort, weil die Nerven der bewussteste, aber auf der irdischen Ebene auch der toteste Teil von uns sind.

Ob dies so stimmt, kann ich nicht  sagen. weil ich mich auf diesem Gebiet zu wenig auskenne. Was an dieser Sequenz des Gespräches für Dich wichtiger zu wissen ist, ist folgendes: Die Eibe erklärt mir hier mit Bildern ein hochkomplexen Zusammenhang. Da ich mich viel mit den Schriften und Vorträgen von Rudolf Steiner beschäftigt habe, nutzt die Eibe hier mein Vorwissen, um mir zu erklären, was sie meint! D.h. ich habe eine Ahnung bekommen, was sie meint, die sehr schwer, bis nicht in Worte zu fassen ist. Wenn Du Dich jedoch nicht mit Anthroposophie beschäftigt hast und über andere Wege forscht, dann würde die Eibe Dir den gleichen Inhalt völlig anders erklären! Deshalb ist es sehr schwierig über solche Gespräche mit Bäumen, Tieren, Pflanzen und Steinen zu berichten, weil diese für jeden Menschen andere Themen und Worte finden würden!

Auch wichtig!: Sie bewegen sich \begin{Large}
für Dich
\end{Large} innerhalb Deiner Glaubensstrukturen, weil sie sich Dir anpassen können. Sie können sich Dir anpassen, weil sie in mit der Ganzheit verbunden sind, allem und Dir und sie dadurch zwei Möglichkeiten haben, die Du nicht hast!

Sie haben keine Gehirn, oder eines das anders funktioniert (Tiere) und müssen/können nicht in festlegenden, missverständlich Worten kommunizieren. Deshalb haben sie sekündlich zu fast allem Wissen eine Verbindung. Allerdings macht es schon Sinn, z.B. einen Baum Baumdinge zu fragen, statt ihn zu fragen wie man läuft, oder Fahrrad fährt, Du weisst, was ich meine\dots

Und sie spüren Dich und zwar ganz! Der Baum weiss also, im Gegensatz zu Dir, in welchen Glaubensstrukturen Du Dich bewegst und wird es Dir so einfach wie möglich machen und Deine Glaubensstrukturen für Euer Gespräch nutzen. Wenn Ihr beide ein gutes, eingespieltes Team seit, wie z.B. meine Pappelfreundin und ich, dann kann der Baum probieren seien Kommunikation zu erweitern.

Das ich mich mit den Bäumen zusammen entschieden habe, diese Dinge aufzuschreiben, liegt daran, dass wir zeigen möchten, wie intensiv, vielfältig und spannend die Welt der Freundschaft von Bäumen und Menschen sein kann! So ignoriere einfach die Dinge, die Dir unlogisch oder seltsam vorkommen, denn die Idee ist nur die Möglichkeiten aufzuzeigen und Dir Werkzeuge an die Hand zu geben selbst mit den Bäumen über Deine, Eure gemeinsamen Themen zu sprechen.

Als ich mich verabschiedete blitzte für einen Moment der Weihnachtsmann in seinem Rot/Weissen Gewand am dunklen Stamm auf.

Auch bei meinem zweiten Gespräch spürte ich sehr viel Respekt vor der Eibe. Unter ihrem Tunnel, den ihre Äste bilden, war es eisig kalt, obwohl die Sonne schien.

Es war eisig kalt, tot kalt.

Die Eibe antwortete: ,,Das stimmt nicht! Der Tod ist nicht gezeichnet durch Kälte! Der Tod bringt ein Zuviel an Hitze! Schau, selbst der, der erfriert hat den Eindruck zu verglühen! 

Du musst verstehen, der Tod bedeutet ein eindringen des Geistes und das geschieht immer mit Feuer/Hitze! Der Körper, die Materie des Körper, die wird kalt. Aber der Tod tritt ein/ bringt der Essenz in Dir Hitze. Deine Essenz wird wieder zu Geist und das kann der Körper nicht aushalten. Er stirbt und der materielle Körper löst sich auf. Der geistige Teil, die Essenz bleibt.''

Die Birs floss an uns vorbei. Die Sonnenstrahlen glitzerten, gleissten auf dem Wasser, das sich breit in der Kurve hinter der Birsfeldener Brücke ausbreitete.

,,Schau!'' sagte die Eibe. ,,Nehme diesen Fluss als Zeitstrom und schaue ihn Dir genau an, spüre ihn genau! Den Zeitstrom.'' Ich tauchte tief in das glitzernde Wasserbild, das in grosser Fülle auf mich zu und an mir vorbei floss. Das Wasser kam auf mich zu! So wie die Eibe mir die Zeit zeigte, kam der Fluss der Zeit auf mich zu und floss an mir vorbei/ durch mich hindurch!

Es war ein überwältigender Moment. In diesem goldenen, reichen Zeitstrom zu spüren, der füllig und kraftvoll seine Wellen durch mich schickte\dots Zeit kommt von vorn!\footnote{Wenn Du weiter zu diesem spannenden Thema forschen willst, dann seien Dir Schriften von Rudolf Steiner, z.B. zum Einstieg im www.anthrowiki.at empfohlen. Aber auch die alten Ägypter z.B. haben sich in ihren Totenbüchern intensiv mit diesem Zeitstrom beschäftigt.}

Zeit kommt von Vorn?

Nicht aus der Vergangenheit Schritt für Schritt in die Zukunft?\dots ! Ouuuih!

,,Schau!'' sagte die Eibe zu mir, die ich mitten in dem Zeitstrom stand und staunte. ,,Für Euch Menschen bedeutet Leben: Im Zeitstrom stehen! Stehen! Nicht bewegen!''

,,Das ist wichtig zu verstehen'' sagte die Eibe zu meinem staunenden Geist: ,,Euer manifestierter, physischer Körper ist schwer wie ein Stein im Wasser! Er bewegt sich nicht! Euer Körper, der wie ein Stein im Strom der Zeit ruht, wird von der vorbei strömenden Zeit berührt und

das

empfindet Ihr Menschen als Zeit!

Die Berührung des Zeitstromes an Dir, der/die Du wie ein Stein in ihm darin bist, diese Berührung, die Dir ein Gefühl von fliessen übermittelt, das ist es, was Du Zeit nennst.

Aber

Das ist nicht die Zeit! Die Zeit ist ein riesiger Strom, der durch die gesamte Bewegung des ganzen Universums miteinander entsteht\dots (der Bewegung der Sterne, der Bewegung der Planeten, der Bewegung der Erde, der anderen Erden, der Kontinente,\dots der Tiere, der Bäume, der Gräser, der Insekten, der Steine, der Sandkörner am Strand\dots )

Wenn Du wirklich Zeit spüren willst, dann lerne in dem Strom zu schwimmen und zu tauchen, natürlich mit Deinem feinstofflichen Körper (Okay, dem einen oder der anderen hilft es vielleicht, wenn er/ sie mit dem materiellen Körper übt/probiert). Du wirst bemerken, das Zeit, wie der Fluss auch ein riesiger gefüllter Raum ist\dots ''

,,Dieser gewaltige Strom fliesst an der physischen Welt vorbei, entlang und diese bewegt sich damit und antwortet darauf

\dots das ist dann auf der Erde Zeit.

Deshalb kannst Du wie eine Ente, eine Gänsesägerin darauf/darin spazieren gehen!

Wenn ich sage Zeit \& Raum gibt es nicht, so stimmt das nur bedingt! Deshalb will ich lieber sagen, Zeit \& Raum sind so riesig, dass Du als Mensch sie nicht als Zeit \& Raum begreifen kannst!

Denn so gesehen, passiert alles gleichzeitig! So wie die Birs gleichzeitig entspringt, fliesst und in den Rhein mündet, der gleichzeitig entspringt, fliesst und in das Meer mündet\dots ''

Dies zeigte mir die Eibe!

Auf meinem Rückweg war Meister Reiher noch da. Ich bedankte mich bei ihm, dass er mich auf die Lektion der Zeit vorher eingestimmt hatte und mir geholfen hatte tiefer in mich hinein horchen zu können, damit ich die Worte der Eibe hören konnte.





\section{Frag' Doktor Weide und Laguz \textarc{[\withlines]l} }

Auf dem Heimweg stand der Reiher noch an der selben Stelle, wie vor dem Rosen- und Eibengespräch. Ich blieb einen Moment bei ihm und er wies mich auf eine kleine Weide hin, die so dicht am Ufer der Birs wuchs, dass ein Teil ihrer Wurzeln ständig im fliessenden Wasser sind.

Sie ist jung und doch gezeichnet, denn sie hat eine grosse, alte Wunde am Stamm, die sie nicht schliessen konnte. Sobald ich mich mit ihr zu einem Plausch verbunden hatte, wurden meine Beine von einem kalten Hauch umweht.

Die Weide strahlte Kühle bis Kälte aus. Sie kann sogar ganz im Wasser stehen, da sie mit den Emotionen, Gefühlen, die das Element Wasser auch verkörpert, so umgeht, dass diese sie einfach durchströmen. Hatte die Eibe zuvor eine starke senkrechte Bewegung in mir gemacht, so strömte der Fluss nun waagerecht durch die Weide und mich durch. Sei hatte eine starke Verbindung zum Wasser aber eine distanzierte. Die Weide kann die wichtigen ,,Dinge'', wie Nährstoffe herausholen ohne vom Wasser davon geschwemmt oder überwältigt zu werden.

Der Reiher, der in der nähe stand, sah mich intensiv an. er zeigte mir, dass das Wasser, wenn ich es als Gefühlssymbol anschaue, bunt wie ein Regenbogen ist!

Ich spürte zurück zur Weide und spürte stark das rechte Auge, dass mir so viele Sorgen gemachte.

Ich nahm wahr, wie in dem Auge ein riesiger Bollen an Emotionen steckte! Diese hatten sich dort angesammelt. es waren alles schlechte Empfindungen, die ich hatte, wenn meine Mutter mir wieder ,,vor Augen hielt'', wie schlecht, faul, selbstsüchtig oder hässlich ich sei\dots Ich hatte, da ich damals in diesen Situationen nicht flüchten konnte, die negativen Bilder, die ihre Worte vor den inneren Augen auslösten, in den Augen gestoppt! Da ich nicht wollte, dass die Worte mich vergiften, hatte ich unbewusst versucht sie an ihrem Entstehungsort zu stoppen. Da ich ein Augenmensch bin, war dieser Entstehungsort die Augen gewesen.\footnote{Das hatte den Effekt gehabt, dass meine Augen, sobald ich eine neue Brille verschrieben bekam, nach einem halben Jahr wieder gleich schlecht schauten, weil sie den Auftrag hatten, die Bilder zu stoppen. Dadurch erhöhte sich meine Dioptrienzahl immer wieder. Später, als ich zumindest begriffen hatte, dass meine Augen bei jeder neuen Brille gleich wieder ,,nachregelten'', liess ich mir nur selten neue Brillen anpassen. Was soweit half, dass meine Augen lange Zeit ,,stabil schlecht'' waren! Verrückt, wie wir uns somatisch versuchen in extremen Situationen selbst zu helfen!}

Zu dieser Zeit trug ich ein Armband das mir der Schamane Amir Oorzhak bei einer Einzelsitzung umgebunden hatte. Ich durfte 30 Tage alles schlechte in das Armband hineinstopfen, um es anschliessend wegzuwerfen. Dort hinein schob ich den Bollen aus dem rechten Auge. Vermutlich wirst Du mir kaum glauben können, wie gut sich das Auge augen-blicklich anfühlte! Aber so war es!\footnote{Schau, wenn ich jetzt folgendes hinzufüge: dann wirst Du mir noch weniger glauben! Aber ich sage es trotzdem, weil es so lustig ist: Ich hatte den Amir-Schamnanen nämlich um zwei Dinge gebeten in der Behandlung. Eines davon war Hilfe für das rechte Auge. Er sagte nichts zu dem Auge und gab mir nachdem er sich um die andere Sache gekümmert hatte das Armband: 30 Tage jeden Abend alles schlechte hineintun. Fertig, Ansage\dots Das Gespräch mit Reiher und Weide fand am 28. Tag satt! Danke, lieber Amir! Und ich habe zuerst gedacht, Du hättest das mit dem Auge vergessen! }

Auf dem Rückweg wurde mir bewusst, das sehen mit den Augen ,,nur'' sehen ist! Die Augen machen nicht mehr und nicht weniger. Und durch die Augen sehe ich, was jetzt gerade um mich herum zu sehen ist!
Nicht mehr und nicht weniger!

Die AUgen können und müssen keine ,,Bilder'' von anderen fernhalten, stoppen, ansammeln!  Das muss, wenn überhaupt nötig mein Hirn schon selber machen. Die Augen können nicht, ohne Schaden zu nehmen, beauftragt werden Gedankenbilder, die durch Worte entstehen, zu filtern. Genauso wenig wie Ohren, Nase und die anderen Sinne nicht wahrnehmen und filtern können/sollten, was mein Gehirn gerne hätte, sondern hoffentlich alles wahrnehmen, was um mich in der Welt alles passiert.

Am nächsten Tag unterhielten wir uns weiter. Mir wurde zu Beginn wieder kalt und meine Herz klopfte spürbar schnell. Die Weide kann das Wasser sehr gut in sich lenken. Sie hat sich auf die ,,Leitungen'' in ihrem Inneren spezialisiert. Sie hat ein starkes Bewusstsein für die Stärke und Beschaffenheit ihrer Leitungen, die in ihrem Stamm und überall in ihrem Wuchs darin sind.

Auf diese Weise kann sie uns Menschen in allem, was mit fliessen und den Leitungen durch die etwas fliesst helfen und unterstützen. wie bei allen Pflanzen macht die Weide dies feinstofflich überall in unserem Körper, wo sie es für nötig hält. Dies macht sie, wenn wir einen Moment in ihrer Nähe verweilen. Sie macht es auch, wenn wir vorüber gehen, nur dann halt sehr kurz. Diese ,,Behandlung'' von Doktor Weide, oder anderen Pflanzen spüren wir nur selten, solange wir nicht bewusst darauf achten. Selbst dann ist es manchmal schwer, genau zu spüren, was der Baum macht. Das die Weide mir so genau noch berichtete, was sie alles macht und ich ess spüren kann, liegt an dem jahrelangen Training! Ja, man musst es üben, es ist wie ein Muskel! Und es ist wie bei allen Dingen, die man machen will, Talent alleine ist hilfreich, aber wer nicht übt, der kommt zu nix!

DIe Weide wäre in der Menschenwelt ein Gefässarzt oder ein Sanitär.Sie kann das Wasser durch die Rohre rein und raus und hindurch leiten. Soe sorgt für einen gleichmässigen Druck und behält aber auch die leitungsdicke/-stabilität im\dots äh, Auge!

So spült sie Klümpchen und Verhärtungen weg. Aber nur, indem sie den Fluss darum leitet. Sie rüttelt nicht daran oder so, sondern erhöht nur leicht den Durchfluss => ,,Brocken'', die den Weg versperren und recht lose sitzen, werden von ihr abtransportiert.

So ist die Weide einerseits sehr wässrig oder besser Wasserexpertin, aber auch fest -> Erde. Denn sonst könnte sie nicht Leitungsspezialistin sein. Ihre Blätter sind schmal und dünn und stark geädert. Die Blätter sind klein, weil die Weide Bahnspezialistin ist und sich viel auf ihre Äste und den Stamm konzentriert.

Zusätzlich spielen die kleinen Blätter leicht im Wind, sind beweglich und flexibel. Das ist ein luftiger Aspekt der Weide, die tatsächlich dem Menschen recht zugewandt ist und gerne hilft, eben wie ein Doktor.

Aus dem Rindenstoff der Weide wurde Aspirin hergestellt, dass als Tablette eine ähnliche Wirkung für die Durchblutung hat. Es wirkt schmerzlindernd. Auch die Aspirintablette kann einen empfindlichen Magen schnell reizen. Die Weidenrinde macht dies noch mehr und hat nicht so eine starke Wirkung. Daher ist die Tablette hier hilfreich.

Am Besten ist es direkt zu einer lebenden Weide zu gehen und bewusst ihre helfenden, heilenden Einfluss zu spüren.

Ein Aspekt des Elementes Wasser ist, dass es uns unendlich viel über Emotionen und Gefühle und den Umgang mit ihnen Lehren kann. Die Weide als Leitungsspezalistin kann auch auf dieser Ebene sehr hilfreich unterstützen und klären. Vor allem wenn es um das fliessen und blockieren von Emotionen geht, aber auch diese in der richtigen Weise fliessen zu lassen, nicht zu schnell oder zu langsam.

Ich besuchte eine zweite, grosse Weide, die nicht direkt am Wasser steht. An ihren Stamm gelehnt mit der tief gefurchten, borkigen, grauen Rinde wird mir gleich warm. Also war die Kälte, die ich bei der kleinen Weide spürte nicht weidenspezifisch.

Deshalb ist es so wichtig viele veschiedene materialien zu untersuchen, mit vielen z.B. Bäumen derselben Art und von unterschiedlichen Arten zu sprechen, damit Du rausfinden kannst, was spezifisch für eine Baumfamilie ist und welches Muster nicht.

Die ,,Strömung'' in der grossen Weide war nicht so stark, dafür die Sonnenstrahlen, die ich plötzlich viel besser in meinem Gesicht spüren konnte.

\section{Der Handwerker Bergahorn und Perthro \textarc{[\withlines]p} }

Am nächsten Morgen war zarter Nebel über dem Rhein und der Matte. Die Pappelfrau, die ich fortan immer als erste begrüsste und die meine Gesprächsmanagerin wurde, schickte mich zu einem kleiner Baum, der kurz über der Erde schon seinen Stamm in fünf aufgeteilt hatte. Die fünf Stämme wirkten wie eine Hand mit ihren fünf Fingern. In der Mitte, wo sich die fünf Stämme aufteilten hatte sich eine Kuhle gebildet. Und in diese Kuhle wuchs ein junges Holunderbüschlein, ein Baum auf dem Baum:

es war ein Bergahorn. Seine feine, senkrecht ziselierte Rinde trägt schöne Moose und Flechten. Er sah im ersten Augenblick belaubt aus, denn er trug an vielen Ästen Samenbüschel. Dunkelbraune, mit dem bekannten Doppelflügel ausgestattete Samen.

Wie die Pappel hat der Ahorn eine starke Verbindung zum Element Luft. Er liebt es zu sprechen. Anders als die Pappel, die sich viel mit sehr ernsten und tiefgründigen Themen beschäftigt, ist der Bergahorn Spezialist für ,,süsse Sachen'' und ein Experte, was die Zahl fünf angeht.

Er ist eine Art Tüftler, Künstler, Handwerker unter den Bäumen und seine Blätter und die fünf Stämme betrachtet er als seine Hände. Der Bergahorn lauscht nicht nur, sondern er ,,bearbeitet'' die Dinge in einer speziellen Form, indem er die Süsse darin untersucht.

,,Behalte die Süss! das Süsse darf man auch behalten!'' sagte er. Worauf er darauf hinwies, dass wir Menschen oft das Bittere behalten. 

Der Ahorn machte mir die (fünf) Geschmäcker bewusst.

Süss: stärkt das Herz und damit die Interaktion, Liebe (gelb) und auch die Freude, Glück (orange). Und! Man darf sie beide behalten!

Salzig: stärkt den Kopf, das Denken. Salz ist strukturiert und kristallin, es besteht aus lauter Würfeln. So hilft salziges zu strukturieren, zu denken und die eigene Persönlichkeit, das ,,Ich'' (R. Steiner) ,,herauskristallisieren''!

Bitter: stärkt Feuer und Leber. Hilft Energie (Auch als Wut) für die Herausforderungen im Leben zu sammeln, zu tragen, zu erzeugen und fliessen zu lassen.

Sauer: zieht zusammen => es bringt mich zu mir, wenn ich mich im Aussen verliere (Hat mit der Eifersucht und den Nieren zu tun)

Scharf: Feuer! Es bringt den Körper in Wallung und in Begeisterung. Wenn das Feuer verdaut wird, kann es Kühlung bewirken indem der Körper schwitzt\dots es stärkt den Körper und den Austausch mit anderen.

Die kleinen Propeller, die Kinder des Bergahorns sind auch kleine Feuerbälle. Samen sind feurig, denn sie enthalten die feinstoffliche Quintessenz einer ganzen Pflanze, eben z.B. eines ganzen Ahornbaumes. Die Samen des Ahorns haben einen starken luftigen Aspekt und damit zu Gemeinschaft und Kommunikation. Das ist seltener. Im Ahornsamen sind Feuer und Luft verbunden und das Feuer bekommt Flügel! Wie kleine Adler!

Es war sehr wohltuend in seiner Nähe. Er stärkte Glück im Bauch und das Herz. Es wirkte alles fröhlicher, freudvoller. 
Rune, meinte er, es müsste eine sein, die aus fünf Stäben zusammengesetzt ist.

Zuhause angekommen, suchte ich also nach einer Rune, die aus fünf Stäben besteht. Ich hoffe, es gibt nicht zu viele\dots ich habe Glück: Es gibt nur eine  Perthro \textarc{[\withlines]p}

Perthro \textarc{[\withlines]p} hat mit dem Schicksalsraum zu tun, sagt der Ahorn. Denn sie hat fünf Stäbe:

2 (Gott + Mensch) dann +3 (Mensch + (Gott + Mensch)) = 5

Damit ist gemeint, als erstes in sich selbst in die Tiefe, Vertiefung eintauchen und dann die Schätze in die Menschengemeinschaft hineintragen.

=> kommunizieren, miteinander interagieren, sprich sich gegenseitig spüren: mit Gott/ mit den Menschen schafft diesen Raum Perthro \textarc{[\withlines]p}.

Wie zur Bestätigung sehe ich zwei Nilgänse in der Birsmündung\footnote{Schau bitte selbst nach: Gänse!}

Die Samen des Ahorns, auf die er sehr stolz ist, wenn ich das so menschliche Gefühl hier nennen darf, verkörpern Feuer und Luft im perfekten Einklang.\footnote{siehe Abbildung}

Das  Feuer, sagt der Ahorn, braucht flügel und es darf nur in kleinen Portionen verteilt werden, damit es kein Flächenbrannt gibt, sondern die kleinen Flämmchen, die Samen, sich richtig entfalten und verbinden können.

Ahorn= Viele kleinen Feuerchen mit Flügeln.

Mit dem Ahorn kommt mir die Assoziation von Leonardo da Vinci und seinen zahlreichen Flugmaschinen, die er gezeichnet und zum Teil gebaut hat. Sie sind ebenso Begeisterungsfeuer, die an verschiedenen Orten wurzelten.

\section{Der Hollenbaum Holunder und Othala \textarc{[\withlines]o} }

An diesem Tag war ich zu einem Menschenspaziergang verabredet. Aber in der Nähe meiner Bushaltestelle, stand ein grosser Holunderbusch. Er muss sich seinen Platz mit eine r Mauer und einem Geländer teilen, was er auf die Art und Weise löst, dass er mehr und mehr darüber, darum und darunter wächst. Er hat eine borkige, faltige Rinde am Stamm und ein Feenlochs! So nennt man Äste, die sich an einer Stelle trennen und an einer anderen wieder zusammen wachsen. Dort, so heisst es, können die Feen durchschlüpfen. 

Feenweg: Gestaltwandlerort, Türaufhalter, Zwischenraum, Gebärmutter (---> Transformiert)

Feen haben Holunderbüsche sehr gerne. Denn der Holunder gilt als Eingangtor zur Anderswelt, zur Feenwelt.

Wenn Du also an einer Hecke oder einem Waldrand vorbei kommst an dem viele Holunder dicht nebeneinander stehen, so bist Du vermutlich an einer Feenhauptstrasse gelandet.

Ich hatte für den Hollebaum nicht viel Zeit. Aber da wir einander bekannt gemacht hatten und verbunden waren,  als ich in den Bus stieg, erzählte mir der Holunder munter weiter auf meiner Fahrt zum Badischen Bahnhof.

(Okay, das ist schon etwas für Fortgeschrittene. Es braucht etwas Training, also mehrere Jahre Training, bis die Aufmerksamkeit so stark ist, dass sie auch auf einer Busfahrt umgeben von vielen anderen Menschen ausreicht, um weiter mit einem Baum zu reden. Ja, ich bin etwas stolz darauf, etwa so wie eine Sportlerin, die jeden Tag trainiert und nun eine gute Zeit macht. Also, Du kannst es auch\dots nach einigen Jahren Fleiss! \footnote{Damit will ich Dich nicht entmutigen, ganz im Gegenteil! Du sollst lieber wissen, dass es seine Zeit braucht.} )

Ich spürte also meine Durchblutung stärker, mir wurde warm dabei. Eine ähnliche Wirkung wie bei dem wunderbaren Holunderbeerensirup\footnote{siehe Rezept im Anhang}

Der Stamm und die Äste des Holunderbaumes haben in ihrem Inneren eine Röhre, die mit einem anderen Material als Holz ausgefüllt ist. Es fühlt sich an und sieht aus wie eine Art feiner Schwamm, oder, für alle die schon einmal einen Filzstift auseinander genommen haben, wie der innere, schwammige Teil, indem die Farbe aufgesaugt ist.

Die Schwamm im Inneren des Holunders hat eine ähnliche Funktion, denn er kann Dinge, Materialien aufsaugen und durch sich hindurch transportieren. Wenn wir bei dem Bild des Filzstiftschwammes bleiben, macht er auch etwas ähnliches: Er färbt zwar nicht bunt, aber er versieht die Dinge mit einer spezifischen Holunderfärbung! Er gibt den Dingen: Sonnenlicht, Wasser, feinstoffliches in seiner inneren Röhre etwas: Holligkeit, Holunderheit, die Bewegung des Holunders, während etwas das rot ist, die Bewegung des Rotes angenommen hat, oder abstrahlt.

Der Holunder ist eine kleine Chemiefabrik, eine Wandelmaschine. Er hat eine sehr enge Verbindung zu Erde und zum Feuer. Diese beiden unterschiedlichen Elemente treffen sich im Holunder.

Die Erde sorgt für den Fruchtbaren Aspekt des Holunders, macht ihn zu einem Mutterschoss

Die Rune Othala \textarc{[\withlines]o} tauchte auf. Sie sieht ein wenig aus, wie ein Mutterschoss. Das Karo und hat zwei ,,Beine''. Das Karo bedeutet bei den Runen aber auch Feuer. 

Diese Rune war übrigens die einzige, die mir ,,auf dem Kopf'' gestellt geschickt wurde. In der traditionellen Runenorakelkeit heisst das, dass dann die negativen Aspekte der Rune im Vordergrund stehen, wenn sie als Orakel genutzt wird.\footnote{Das sie auf dem Kopf stand, bemerkte ich dann zuhause, als ich die Rune nachschauen ging, da die Runen ja neu für mich sind.}

Weiter plauderte der Holunder, er/ bzw. er redete von sie, sie wäre durchlässig und würde dennoch allem eine, nämlich ihre Färbung geben, denke nur an die schönen, schwarzen Beeren\dots sie berühre die Dinge

mit ihrem Geruch, ihrem Feuer. das Bild der Frau Holle tauchte auf. Der Urmutter, der Stammesmutter: Rund und faltig von den vielen Schwangerschaften. Mit grossem Busen und einem verschmitzten, runden Gesicht mit einem breiten grinsen. Die Alte, die lebt und gelebt hat und gerne mit einem Whisky in ihrer tiefen, rotlehmbraunen Höhle unter der Erde wohnt. Baumwurzeln ragen hinein und erzittern, wenn sie aus ihrer Zigarre eine dicke,Qualmwolke ausstösst und zufrieden lacht.

Ihr Geruch, der Hollergeruch hat etwas von der Alten: Er ist nicht gar so fein, wenn Du eine  tiefe Nase nimmst:

Eher so nach Kompost, Verrotten, Pipi, wärme, Ver-weseung und dennoch wie Honig und süsslich, wie richtige Grossmütter riechen\dots

Aber

sie gibt den Dingen eine geistige (feurige) Komponente

=> Ver-wesen = trennen der Dinge

in physisch und geistig

und => dadurch bekommst Du einen besseren Blick auf beide Aspekte!

Einen freien Blick, ohne Urteil

Rausch + Ekstase (Holunderschnaps) sind ihre Methoden

Die Verbindung zwischen

Materie--- => + Geist--- 
	
= Emotion--- tritt ebenso deutlicher hervor

\dots Ja, so ein Holunder kann fest viel plaudern auf einer 5 Minütigen Busfahrt. Dann kam der Badische Bahnhof und dort brauchte ich meine Aufmerksamkeit für den Tram- und Strassenverkehr und dafür den richtigen Zug zu erwischen!

Auf der Fahr zu meiner Menschenfreundin wurde mir bewusst, dass ich keine Ahnung von den Runen hatte! Ich stellte zuhause fest, dass ich die Rune Othala auf dem Kopf geschickt bekam, wusste aber nicht, ob ich sie falsch interpretiert hatte, weil ich meinte, sie gehöre so, oder ob sie mir von dem Holleholunder verkehrt herum geschickt worden war, weil die Runen auf dem Kopf eine andere Bedeutung haben.

Es ist nicht so einfach mit Werkzeug zu schaffen, das man noch nie benutzt hat.

Ich hatte sehr platte Füsse von der Wanderung, als ich nach Hause kam. Aber das machte nix. Denn ich hatte dafür ein lekka Stück Nusstorte mitgebracht vom Ausflug und schürfte einen Punsch dazu. Eigentlich wäre eine kleine Pause\dots

Doch es liess mir keine Ruhe und ich schnappte die Torte am Compi und vergrub mich wieder in den Runenseiten, die jede etwas anderes spannendes erzählte.

Mit den Runen, die ich bisher kennengelernt hatte und den Bäumen, die sie mir gezeigt hatten, wurde es schwierig. Denn es war nicht so ,,einfach'' wie bei der Eibe, die ihre  persönliche Eibenrune hatte und mit mir darüber sprach. die Runen, die mir die anderen Bäume schickten, liessen sich nicht alle einfach als ein spezifischen Teil des Baumes anschauen. 

Einige ja, andere nur am Rande. Ich versuche hineinzuspüren und mir dämmerte wieder herauf, welche Zeit war. es waren Rauhnächte, die Zeit der Orakel. Das konnte erklären, warum einige Runen nicht eins zu eins zu den Themen der Bäume Passte, die sie mir schickten\dots

Ich beschloss noch mehr daran zu arbeiten, nicht gleich ein muster erkennen zu wollen, nicht gleich eine Rune dem Baum wie ein zu enges Jäckchen herauszuquetschen, wie Frau Seppel den armen Mopsi in einen Hundepullover.

Unser Gehirn liebt es, das zu machen und oft ist es gerade im Alltag sehr praktisch, Du kannst da oben im Denkstübchen Sachen managen, ohne wirk-lich zu denken. Aber bei den feinstofflichen Dingen geht das nicht, da musst Du wissen, was DU angegooglet hast und musst wissen, wann Du eine Vor-Stellung hast und wann Du es mit ,,echten Baumworten'' zu tun hast, weil eben beides Dir so erscheint, als wäre es in Deinem Gehirn auf der gleichen Leinwand.\footnote{Ich muss hier erwähnen, dass genau diese Sache auch ein längeres Training braucht. Denn es gilt nicht nru Absicht und Aufmerksamkeit für das Feinstoffliche in die eine Richtung zu trainieren, sondern generell etwas genauer hin zuschauen, was da genau gerade in Deinem/meinem Kopf spricht/Emotionen auslöst, etwa mitteilen möchte\dots} 

Was hatte Frau Holle mir gesagt: ,,Hab' Spass!'' AN diesem, dem Silvesterabend brauchte ich etwas Zeit um herauszufinden, was das bedeutet. Denn mein Mann und ich waren bei Freunden zur Party eingeladen. 

Ich musste mich entscheiden. Baumwelt/Menschenwelt. Feinstoffliche Welt noch tiefer spüren/ materielle Welt mit Freunden feiern, aber auch viel Essen und Trinken, lange aufbleiben\dots

Es war sehr schwer, denn natürlich gehe ich gerne auf Feste. Aber ich wollte keine halben Sachen machen. Nur ein wenige feiern und dort früh ins Bett, oder so.

Rauhnächte sind nur einmal im Jahr. Es ist ein spezieller Raum, der genau in dieser Zeit offen ist. Wir haben durch die Materie, durch diese Verzögerung, Verlangsamung des feinstofflichen in die Materie hinein, bestimmte Rhythmen auf der Erde. Dazu gehören die -Jahreszeiten, aber auch ein ige spezielle Zeitfenster, die jedes Jahr stattfinden.\footnote{Nur, weil wir tatsächlich oft keine Ahnung mehr haben, heisst dass eben nicht, dass die alten Lichtfeste, oder Ostern, oder Rauhnächte nicht dennoch stattfinden. Vielleicht würden sie eines Tages verblassen, wenn selbst der letzte Mensch sie nicht mehr bewusst anschaut. Aber heute sind wir, Gott sei Dank (Ja, da war auch noch etwas\dots ), noch lange nicht an diesem Punkt!}

Ich entschied mich für die Baumwelt. Es war wie gesagt nicht einfach, denn ich hatte da meine Prinzipien, die musste ich über Bord werfen. Und ich musste aufhören mich zu vera***en. Ich musste aufhören mir einzureden, mir irgendwelche Gedanken darüber zu machen, das ich nun mit Bäumen reden wollte!\footnote{Du kannst Dir nicht vorstellen, wie dumm ein Hirn machen kann, wenn Du einfach mal ganz gechillt mit Deinem Baumfreund reden willst!}




\section{Der hoheitsvolle Buchenkönig und Elhaz \textarc{[\withlines]R} }

Am Altjahrestag lerne ich die Buche kennen. Sie ist natürlich schon vorher da, aber heute werde ich von der Pappel zu ihr geschickt, zu ihm geschickt\dots

Die Buche zeigt mir die Rune Elhaz (oft Algiz genannt) \textarc{[\withlines]R} als Rune für das Jahr.

Die Buche selbst fühlt sich sehr erhaben und königlich. Sie ist verbunden mit dem germanischen Gott Tyr.

\begin{Large}
,,Blutbuche, Bitteschön!''
\end{Large} 
dröhnt es. Denn Blut ist wichtig! Blut ist, was ich bin!

Blut---> Ich-Behauptung = Be-haupt-ung = sich be-wusst sein (Im Sinne von: um/von sich wissen!)

Die Blutbuche ist kein Baum von vielen Worten

Dünne Rinde, hartes, rotes, schönes Holz ---> Luft/Feuer und Erde ---> viele Wurzeln

Ich frage schüchtern nach Wasser/Emotionen\dots

Gefühle: Birs/Emotionen weit weg = uninteressant für Ich (Kern)

Okay?!

2 Krähen sind die ganze Zeit in der Nähe, um mich herum\dots Hugin 1 Munin\footnote{Aus Wiki: Hugin (= isl.) von altnord. Verb ,,huja'' = denken; von ,,huji''= Gedanke, Sinn/ Munin (=isl.) von altnord. Verb ,,muna''= denken an, sich erinnern}

Der Buchenbaum ist ni\dots der Blutbuchenbaum ist noch nicht fertig mit seinem Stakkato:

stark, fein, licht, kräftig, glatt, ---> Buchstabe

Baum = königliche Persönlichkeit

Innerer Raum: Palast, Königsaal= Gerichts- und Regierungsraum, 

Tradition, Gesetze, Benimm,

Hierarchie, Gerechtigkeit, Führung

= Kraft, Stabilität

Meisen streiten = unwichtig (Die kleinen Piepmätze!)
Hund: Hugin und Munin fort

\section{Weitere Fragen an Doktor Weide und Laguz \textarc{[\withlines]l} }

Doktor Weide, der Grosse, Leitungsspezialist mit Ein-Fluss hilft mir meine Leitungen im Inneren zu reparieren\footnote{Siehe Anhang}. Wenn er das macht, dann hat es natürlich eine Spezialwirkung, weil er ja ein lebendiger Baum ist und die Weidenwirkung dazu kommt.

Wichtig: Auf der körperlichen Eben, kann ich nicht ,,besetzt'' oder ,,beschmutzt'' werden (psych./ phys. Gewalt) = unmöglich!

Denn: Alles, was in mir ist, jeder Körperteil, physisch und feinstofflich = Meins!

=> Nix Dreck oder Böses, etc. von jemandem anderen ,,reingetan''!

=> Nichts muss beseitigt werden! Alles darf an seinem Platz bleiben.

Es kann verspannt, verkrampft, verknotet, geschwächt sein, etc.

Eine Ausnahme: Schlechtes Essen!

---> Schlechtes/Ungesundes Essen wird dem Körper, physisch und feinstofflich, mit der Zeit schaden! Und zwar, weil es schlicht in den Körper gemampft wird und sich zu diesem umbildet! Unser Körper bildet aus dem Essen neue Zellen (physisch und feinstofflich) und er kann nur das nehmen, was wir ihm an ,,Baumaterial'' zur Verfügung stellen.

Aber, alles, was ich nicht physisch über meinen Mund, die Haut, die Nase, also direkt mit dem Körper aufgenommen habe, ist nicht in mir!

=> Mein Körper = alles Meins! Inklusive dem Platz, den er einnimmt!

,,Und das Bändeli vom Amirschamanen?'' frage ich. ,,Hat einen psychologischen Effekt! Das ist eine andere Ebene. Dort kann es das Gefühl geben, dass negative Dinge an mir kleben, die dort nicht hingehören und auf dieser Ebene kann ich die Dinge z.B. auf diese Art mit dem Bändeli entfernen.

Zu Laguz \textarc{[\withlines]l} = Schlange (auch)

---> tanzt, bewegt sich => Tip: Wirbelsäule sanft und stetig geschmeidig halten!

Frag' Du nur Doktor Weide. 

\section{Die schöne Rose hat Stacheln und Fehu \textarc{[\withlines]F} }

Am ALtjahrestag zeigte mir die Rose ihre Stacheln besonders deutlich, sie stachen aus ihr heraus. Als mich mit ihr verband, spürte ich die Stacheln sehr stark und empfand es als unangenehm

Ich dachte, sie lässt niemanden an sich heran,\dots und als Mensch empfand ich das als sehr unfreundlich.

,,Doch?'' rief da die Rose. ,,Die Bienen!'' und wenn sie einen Mund zum grinsen gehabt hätte, hätte sie es jetzt geatn. ,,Die Bienen, die lasse ich wohl an mich heran. Meinen Liebsten lasse ich ganz dicht zu mir und ich mache ihnen die schönsten, duftensten, süssesten Blüten weit und breit!'' Lächelte sie.

,,Mache es wie ich, lasse Deine Liebsten, den Partenr, der genau zu Dir passt an Dich heran! 

Und der, der angetrampelt kommt und an Dir frisst oder Dir die Zweige knickt und sonst nichts gescheites weiss:

---> Der spürt die Stacheln!

,,Ich schenke mich denen, denen ich schenken kann! 
 
Du darfst nicht nur, Du solltest sogar Stacheln haben!

Sie sind erlaubt!

Und sie machen/schaffen einen Wert in der Beziehung! Das ist auch ein Aspekt von Fehu \textarc{[\withlines]F}

Rose strahlt und sagt,

weil sie mich mag, hat sie mir die ersten tage eben nicht so ihre Stacheln gezeigt, sondern ihre Hagebutten!

Und heute war es Zeit für die Stacheln. Wenn Du meine Stacheln spürst, dann können sie für Dich zu Akkupunkturpunkten in der Landschaft werden! Lauter kleine fokussierte Wahrnehmungsorgane! Sie steigern Deine Aufmerksamkeit!

Und Fehu \textarc{[\withlines]F} bedeutet auch sich zu verteidigen!

Am Neujahresmorgen raunt die Rose, als ich recht verfroren an ihr vorbei husche: ,,Zum Schönsein gehört sich Raum nehmen, sich zeigen, sich präsentieren! Und nicht nur bei einem Fotoshooting solltest Du DIch ins rechte Licht rücken!''


\section{Die Eibe erfindet eine Klapprune und Teiwaz \textarc{[\withlines]t} }


Altjahrestag bei der Eibe. Eine Gänsesägerin taucht auf. 

Gänsesäger sind Meister im tauchen. Sie fangen im spürudelnden, fliessenden Fluss Fisch. SIe könnenblitzschenll sein über und unter wasser, sie wissen mit der Strömung in rasender Fahrt den Stro, runter und mit pashcenden Füssen und tüchtig flatternd den Strom rauf zu reisen. Gänsesäger können Dich mitnehmen in jeden Winkel Diener inneren Emotionen und Dir die silbrigen und goldig schillernden Reichtümer, in Dir zeigen\dots wenn Du Dich nicht vor Dunkelheit und schmoddrigen Stellen fürchtest, die für die Fische/ Dein Gefühlsreichtum am nahrhaftesten sind.

Sie also, Frau Gänsesägerin taucht auf im Wasser vor der EIbe. Die Eibe zeigt mir ihren Inneren Raum. Er ist kein gebäude, wie bei der Pappel und der Buche\footnote{Nochmal, weil es so wichtig ist, jeder Mensch und jeder Baum teilen eine eigene Freundschaft. Die Dinge, die ich sehe, kannst DU ähnlich spüren, musst es aber nicht! D.h. auch meine Wahrnehmungen können Dich unterstützen Deine Wahrnehmungen zu schärfen, aber nur, wenn Du Deine eigenen Wahrnehmung selbst suchst! Ruhe DIch also nicht auf meinen aus!}

Der Eibenraum: der dunkle, unendliche, mächtige und fruchtbare Wald! Der Wald des wilden Mannes.

=> Dieser Wald ist nicht nett!

Er ist angefüllt mit allerlei KReaturen, er ist dunkel und dicht!

Wo ist der Raum? Der Raum, oder Linien, DInge, die ins Unendliche gehen nach oben/unten?

Er verschwindet nicht einfach in der Leere, im Nichts, auch wenn uns das so scheinen mag, weil er unendlich und damit unüberschaubar ist\dots

Dieser unendliche Raum ist, wie Dein eigener auch, angefüllt!

Angefüllt mit dem Zauberwald ---> Mit der Anderswelt? \footnote{Der Ort, wo sich die Feen und Gnome und all die anderen Wesen aufhalten\dots}

Die Eibe zeigt mir einen kleinen Zaubertrick: Hokuspokus, aus einer Rune Eiwaz mach, klippklapp, wenn si in der Mitte waagerecht umgeklappt wird\dots eine neue Rune: Der Pfeil, Tiwaz \textarc{[\withlines]t}\footnote{Siehe Abbildung}

,,Schreib's auf!'' sagt die Eibe!

Tiwaz gehört zm Gott Tyr und Tyrs Baum ist die Buche. Doch in diesem Fall ist die Reihenfolge relevant für das Jahr 2020: erst Elhaz (\textarc{[\withlines]R} dann Tiwaz \textarc{[\withlines]t}

Und dann wollte die Eibe ja auch ihren Runenzauber vorführen, der später noch wichtig werden wird.





\section{Der Eichenmann das Feuer Ingwaz \textarc{[\withlines]\ng} und sein Pferd Ehwaz \textarc{[\withlines]e} }

,,Nimm die Rune mit dem Pferd!'' sagt später die Eiche, die neu hinzugekommen ist. Die Rune mit Pferd ist Ehwaz \textarc{[\withlines]t}. Sie wird die dritte im Bunde der Jahresrunen sein.

Die Eiche selbst hat es mehr mit der Rune  Ingwaz \textarc{[\withlines]\ng} Sie ist die Rune des inneren Feuers. Die Eiche ist verbunden mit: Erde + Feuer = Erdfeuer

Sie kennt sich aus in dem Bereich--->

Ich + Erde

|

Beziehung von Ich + Erdfeuer\footnote{Jeder von uns hat ein solches Feuer in seinem innersten Kern. Es sorgt für die Be-geist-erung die Ideen aus unserem innersten Kern auf die Handlungsebene nach Aussen bringt}

Ich blicke auf den Fluss und frage die Eiche, wie ihr Wirkungsbereich in Beziehung zum Wasser/Emotionen steht. Sie/er sagt sie fliessen amm Kern vorbei, wie bei der Buche. Sie sind unwichtig für Ich.

Die Eiche erinnert mich an das Eichenholz, das Mooreiche genannt wird. Es ist kohlrabenschwarz und matt glänzend und hat viele hundert Jahre an einer morastigen Stelle überdauert ohne zu verrotten => Substanz, damit ist der innere Kern gemeint, überdauert.

Der Raum der Eiche: Feuer + Erde = der Raum der Erdschmiede/ Gnome

Sie/er hat einen Wilder-Maa-Anteil => dem ,,wilden'', wilden Mann, dem fruchtbaren Teil. das erklärt, weshalb ich seit ich begann mit der Eiche zu sprechen ein sehr starkes Lustempfinden im Unterleib spürte. 

,,Ja, genau dort! Hab Spass!'' ruft die Holle vergnügt vom Holunder herüber. Sie ist die Partnerin des wilden, wilden Mannes!

Indem Augenblick ging ein Mann mit einer Bierdose unterhalb an der Eiche auf dem Weg vorbei. Er hörte nicht auf zu glotzen! Und dann lächelte er! Ich geriet in Panik, obwohl mir schon klar ist, dass ich ja sichtbar da an den Bäumen ,,herumhänge'' und z.T. sehr an der Schamgrenze merkwürdig aussehen könnte, hat mich der Blick erschreckt.

Denn dieser Mann, der schaute auf diesen Lustteil. Und damit auf einen Teil von mir, während die meisten anderen Leute wegschauen oder ihre Glaubensstruktur über ,,Menschen, die mit Bäumen reden'' anschauen. Für diese bin ich, mein Kern, unsichtbar.\footnote{In einem menschlichen Gespräch kann es sehr unangenehm sein, wenn die Person nicht mich, sondern nur ihr Bild von mir anschaut. Beim spüren in der Stadtlandschaft kann es hilfreich sein, weil die Blicke dann wie durch einen hindurchgehen, oder an diesem Bild wie abprallen und nicht bis zu mri durchdringen.}

Also, dieser Typ schaut mich an! Und er schaute auf den Frauteil und ich wollte nicht, dass er das macht! (Er war nicht ,,mein'' Typ)

Nach einem kurzen Moment der Verwirrung und Panik und einer Beruhigung durch Fluchtgedanken, spürte ich, dass genau dies das Thema war, das mir die/der Eiche zeigte: Männliche Lust. Genau: mit männlich\dots 

,,Muss ich mit arbeiten\dots '' dachte ich zweifelnd.

Frau Holle aus dem Holunderbusch: ,,Tsss! Arbeiten!\dots Spass haben damit!'' Und sie lachte rau und tief\dots

Ja, es ist möglich einen roten Kopf zu bekommen, wenn Du mit den Hollebäumen sprichst und mit der Frau Holle direkt verbunden wirst\dots

Am nächsten Tag bin ich untreu und suche mir eine Eiche in der Nähe, aber weiter weg vom Weg. 

Sie sagt: ,,Runen sind IM inneren Raum 

+ sind Raum  ---> Wie die Rune Ingwaz \textarc{[\withlines]\ng} ein Feuerraum ist im Inneren.\footnote{Sowohl im inneren des Baumwesens als auch in jedem Menshcen}

Sie zeigt mir die Rune IsaDie Bäume haben die Sprache der Runen erfunden, bzw. die Bäume haben mit den Runen Räume geschaffen in die Bäume und Menschen gemeinsam gehen können!

Diese Räumen haben mit der Rune Ingwaz \textarc{[\withlines]\ng} zu tun, dem Feuerraum im innersten Kern, sowohl bei einem Baumwesen als auch im Mensch.

\section{Die Eiche beschreibt den Unterschied zwischen Eis, Isa \textarc{[\withlines]i} und Wasser, Laguz \textarc{[\withlines]l} }

Siezeigt mri die RUne Isa, Eis \textarc{[\withlines]i}. SIe ist einfach ein senkrechter Strich/Stab.

Ein Haufen Hugin/Munins versammelt sich in den umliegenden Baumkronen und fliegt über uns und ruft. ,,Oben, oben!'' rufen sie. Denn ihr Ruf öffnet in der Tat den oberen Raum.

Ich wechsle den Baum und beginne ein Gespräch mit der alten Eiche. Sie zeigt mir verschiedene Ebenen in denen Runenräume entstehen können \footnote{Siehe Abbildung}, vorhanden sind, je nach Ebene des Themas mit dem ich die Rune in Verbindung bringe.\footnote{Geistige Räume sind da, wenn ich die Aufmerksamkeit darauf richte. je öfter ich das tue, umso länger bleiben sie dort. Aber es ist wichtig sich zu merken, der Raum ist da, wenn Deine Aufmerksamkeit dort ist und wenn sie nicht dort ist, dann ist er weg!}

Sie macht einen Sprung, ja so ist das, je nach Baum, geht es zackzack von einem zum nächsten:

 Isa \textarc{[\withlines]i}= Eis= starr, weil keine Richtung angegeben ist (kein Häckchen)

Laguz \textarc{[\withlines]l}= Wasser= sehr bewegt, denn das Häckchen zeigt Bewegung an

so kann Laguz \textarc{[\withlines]l} z.B. auch eine Schlange symbolisieren oderr Wirbelsäule, die ja auch beweglich ist.

Hier nimmt die Eiche Bezug zu dem, was die Wiede mir gesagt hatte zu meinem Rückenproblem. Die Bäume wissen, was der andere Baum mir erzählte, weil ich ihnen dieses Wissen mitbringe, weil die Inormationen in mir enthalten und für die Bäume lesbar ist. Das ist serh praktisch, denn auf diese Weise ist es einfacher für den Menschen die Sachverhalte zu verstehen, weil jeder Baum Bezug nehmen kann auf seinen Kollegen.

Eiche: = mit Gott verquickt

einem fruchtbaren Gott

er kennt sich aus mit Reichtum ---> Gnomen

\section{Der wilde Maa und Frau Holle über den Reichtum Fehu \textarc{[\withlines]F} und Blut}

Jetzt aufgepasst: Finanzberatung de l'homme sauvage:

$Gold/Reichtum( feinstofflich) \neq Erde, Materie$

Gold = Energie

=> es entsteht durch Bewegung, bzw. muss bewegt werden, um zu ,,wirken'', um zu wachsen

aber 

es ist trotzdem Materie

$\to$ entsteht aus/ durch Bewegung von Materie

$\Rightarrow$ ,Banken-Kopf-Geld' = nicht da! Oder wertlos

$\to$ Finanzblase

Rune Fehu \textarc{[\withlines]f} lehrt über Geld/Reichtum
$\to$ Rose ist Spezialistin für Reichtum

Gold/Geld kannst/musst Du stricken/weben 
$\to$ wie die Gnome es machen

$\Rightarrow$ wenn der feinstoffliche Körper schwingt, webt = Lust

Lust = handeln $\Rightarrow$ Energie $\Rightarrow$ wie Überschuss

Lust $\Rightarrow$ Handeln/machen $\Rightarrow$ Materie/Erde

er-greifen  $\to$ erzeigt Energie = Gold

wenn in der Materie

$\Rightarrow$ Nicht auf Geld/gold warten

Nicht das Geld ,,ver-dienen'' 

$\Rightarrow$ da kann nicht viel rauskommen, sondern:

Geld/Gold = Vorgang= Bewegung! Nämlich: Materie mit Lust ergreifen!

Lustvolles, vertieftes Tun = Geldstricken

Wichtig: Immer im Bereich der Materie

$\to$ Es können anfänglich lustvolle Gedanken + Ideen sein,

jedoch

Endergebnis = Materie

dann = Energie (gebunden an Materie und Lust)

Lust = Physisch

Liebe = feinstofflich

und beide zusammen= Bewusstsein = Liebeslust = Reichtum

anders: Reich-tum $\Rightarrow$ Reich sein = Energie/ Geld halten können:

entsteht durch das Bewusstsein vom eigenen (inneren/feinstofflichen) Raum

und natürlich Liebe, wobei Bewusstsein = Liebe = Reich-tum

$\Rightarrow$ der innere Raum ist gross genug! (Ist er es?)

$\Rightarrow$ Person $\neq$ ,lieb', liebevoll, gut sein, etc. 

sondern sie muss einfach in ihrem Inneren Raum genug Platz haben $\neq$ Mangel!

Reichtum behalten: Der innere Raum muss Platz dafür haben! Mein Reichtum auch der materielle muss in meinem inneren Raum Platz haben = dick genug sein, innerlich,

dann

$\neq$ dicker, physischer Körper!

$\Rightarrow$ dick sein = Mangel an innerem Raum\footnote{Das kann bedeuten, ein Mensch, der dick ist, traut sich nicht seinen ganzen feinstofflichen Platz einzunehmen. Er zieht seinen inneren Raum ein, z.B. um sich zu schützen, und da dieser dann zu klein ist entsteht ein Mangel, der sich darin äussert, dass der Mensch grobstofflich zunimmt, siehe Abbildung . Nicht gemeint ist, dass ein dicker Mensch keine Tiefe im Sinne von Gefühl/Intelligenz oder ähnlichem hat! es geht hier um den feinstofflichen Raum, der für jeden Menschen im und um den Körper für jeden etwa gleichviel zur Verfügung steht.}

\section{Der Hollenbaum Holunder der Innere Raum, Perthro \textarc{[\withlines]p}}

Der Hollebaum, der mit mir die Rauhnächte teilt und Frau Holle einen gemütliches Lehrzimmer bietet, ist ein Experte im Raum schaffen. Er steht gleich neben einer Mauer auf der ein Geländer steht. Er ist darüber hinaus und hindurch gewachsen. Stück für Stück.

,,Beschränkungen sprengen!'' meint die alte Frau Holle dazu. Eigene und fremde\dots

Dann tönt sie wie die Rose, als sie sagt: ,,Die eigenen Blüten \& Früchte, das Eigene darf und muss Du anderen ,zumuten'! Du hast keine andere Möglichkeit, denn Du bist ja da! Du kannst Dich nicht verstecken.''

Der Saft der Holunderbeeren mit seiner Tiefdunkel violetten Farbe hat einen engen Bezug zum Blut. das Blut, unser Lebenssaft, trägt unsere innereste Prägung, unser ureigenes Muster in jedem Tropfen. Die Holunderbeeren wollen uns daran erinnern, dass dieseeureigene Muster in uns ist und genau so richtig ist. Wir können es nicht verstecken und wir können es keinem anderen Menschen aufstempeln, aber wir können unserem Leben, unserer Umgebung eine wunderschöne, krfatvolle, tiefviolette Färbung geben!

Ich fragte nach dem Wasser. Die Birs fliesst an uns vorbei, sie schimmert frisch im Morgenlicht und bringt eine leichte Biese mit, die das Flussbett für eine freie, unbeschwerte Fahrt nutzt.

,,Wasser wird durch den inneren Raum eingefärbt. Alles, was ich in mich hineinnehme wird durch mich gefärbt! Wenn es durch mein innerstes gegangen ist! Und in diesem innersten Verwandlungsbereich, in diesem Zwischenraum, dort ist der Raum von Frau Holles Kessel Perthro \textarc{[\withlines]p}. Du kannst Dir Perthro vorstellen als den umgekippten Kessel der Perchta, der FRau Holle.''\footnote{Perthro $\leftrightarrow$ Perchta. Bin keine Linguistin. Finde es schön, dass sich die Wörter ähneln. ABer schön ist schön $\neq$ wissenschaflich\dots muss es?}

,,Und so Raum und Geraum, Geruch = leicht Humus, Verwest\dots Humus =  beschleunigt neues Leben, frag uns Pflanzen\footnote{Wir sind keine Veganer}

$\Rightarrow$ ewiger Kreislauf 

\section{Frau Holle und das Jahr, Jera\textarc{[\withlines]j}}


dieser Kreislauf entsteht durch das Gewicht der Materie. Feinstoffliche DInge befinden sich ausserhalb von Zeit \& Raum, bzw. sind eins damit\footnote{Siehe Eibe und Zeit}

Doch alle Wesen, die an Zeiträume gebunden sind, weil sie grobstofflich sind, die bestehen in Kraisläufen. Einer dieser KReisläufe, die für die Pflanzen und damit für alle Wesen sehr wichtig sind, sind die Jahreszeiten.

Die Rune Jera \textarc{[\withlines]j} ist der Raum für die irdischen Kreisläufe: Jahr, Jahreszeit, Rhythmus, Monat. Mond, Woche, Tag, Stunde,\dots Zeitpunkt und damit auch von Ernte, Belohnung\dots

,,Aaahh, und vergeiss nicht!'' Gluckst die Frau Holle: ,,Humus ist hat einen Anteil an Reichtum = Gold/Geld + Strickprozess!''

,,Aber'' sagte sie und schaut ernster drein: ,,Der Strickprozess des Reichtums ist mit Lust verbunden. Und Lust, so heisst es heute, und ,es ist wichtig, zu wissen, was -es- heute heisst, damit DU weisst, was Du denkst! Und Lust, so heisst es heute $\neq$ weiblich!

$\Rightarrow$ Lust $\neq$ weiblich, des halb $\neq$ richtig, wenn Frauen etwas anderes stricken als Socken!''

,,Denke nicht daran, zu glauben, dass das auf Dich, wenn Du ein Mädchen bist, nicht zu trifft!'' Frau Holle, die Frau mit dem Goldtor: ,,Ich kenne mich da aus Mädchen, was meinst Du, weshalb haben sie die Hölle nach mir benannt!'' Ein gurgelndes Lachen folgt, vermischt mit feuchtem, tiefen Husten\dots

,,Spass bei Seite! Habe Lust, Maidli, habe Lust! Schaue nur, schaue, mein Maidli, dass Du lernst Deine Lust als Dien Reichtum zu betrachten! Deinen Reichtum, den nicht jeder sehen muss, den nicht jeder kennen muss, Deinen Reichtum , den Du da und dort teilst, wo Du getragen bist! Und wenn DU stark wie eine kleine Mutterkuh\footnote{nocht wie ein kleiner Ochse} geworden bist, dann spielt es keineso grosse Rolle mehr! Hüte Deine Lust, wie ein Schatz! Sie ist es, Dein Grösster!''


\section{Der Osagekrieger und sein Pfeil Thurisaz \textarc{[\withlines]\th} }

Kommen wir zum Riesen, Hammer, Dorn Unwetter zu Thurisaz \textarc{[\withlines]\th} und zum Osagekrieger. 

Ich verbinde mich mit dem Baum. Der ist erst einmal etwas beleidigt: ,,Sch*** Name! Milch-Orangen-Baum! Ich fasse es nicht!'' Nein, er mag sicher in seinem Namen nichts von Milch, wie von der Kuh und Orangen, alles Gemüse, wissen.

Der Osagedorn (Oder bei uns eben Milchorangenbaum) kommt aus den USA. Dort diente er mit seinem Holz den Osagekriegern für den Bogenbau. Zwei Möwen kreisen über uns. Die Früchte des Osagedorns sind hell-kräftig grün. Sie sehen aus, als hätte Mandelbrot bei ihrem Design mit geholfen, sie sehen wie Fractalbälle aus. 

Unter seinen Blättern versteckt, trägt der Osagedorn lange, kräftige Dornen. So wird er vieler Orts als lebendiger Weidezaun gesetzt, da er als dichte Hecke gezogen nichts hinein- und nichts hinaus lässt. 

Ich spüre die Vielen Dornen des Baumes wie Sinnesorgane. Jeder Dorn ist wie ein punktueller Minisauger, der seine Umgebung aufsaugt. Der Baum ist ein super Beobachter. Die Aufmerksamkeit der einzelnen Dornen, geht einige Meter weit. Für mich fühlt sich der Baum an, wie ein Strahlenball, oder um im Bild seiner Früchte zu bleiben, wie ein Fractalball, der Strahlen von Aufmerksamkeit ausstrahlt und seine Umgebung wahrnimmt.

Die Dornen verbinden den Osagedorn mit dem Luftraum. und die Emotionen bleiben daran hängen. Er nimmt andere Wesen sehr gut wahr, manchmal, wenn sie zu dicht kommen, dann spüren sie ihn auch: Piiicks! Meistens schmerzhaft!

Das Holz, das für den Bogen benutzt wird und neben dem Eibenholz sehr begehrt ist. Das Holz hat feinstofflich die Fähigkeiten der Dornen, die es an den Schützen weitergeben kann, wenn er bewusst mit seinem Bogen verbindet. Dann kann dieser dem Schützen helfen sich zu fokussieren und punkt-picks-tuell sein Ziel anvisieren.

Der Osagedorn spricht viele unsere Sinne stark an. Seine Früchte riechen nach Orange und Seife. wenn sie aufgeschnitten werden, kommt eine bittere Milch heraus. Daher der Name Milch-orangen-Baum. Der Osagekrieger macht wach, manchmal unangenehm mit dem intensiven Geruch oder seinen Dornen\dots

Der Osagedorn ist eng verbunden mit dem Kriegertum und der Jagd. Er ist ein Jagdbaum, ein Kriegerbaum, ein männlicher Baum\dots\footnote{Don't forget. bei solchen Aussagen handelt es sich um Symbole, Bilder! Diese können und sollen Dich unterstützen das ureigene Muster dieses Baumwesens zu verstehen. Doch die Bilder sind nicht der Baum! So wie ein Foto von Dir, einen realen Moment von Dir zeigt, aber nicht Du ist!}

,,Milchorangenbaum, tsetse\dots, so ein doofer Name!'' sind seine Abschiedsworte an dem Tag!


\section{Die weisse Jungfrau Birke kommt mit Hagel, Hagalaz \textarc{[\withlines]h} und Berkana \textarc{[\withlines]B} }

Der Neujahrestag ist lang. Am Nachmittag gehe ich wieder  raus, es sind noch soviele bäume und RUnen zu treffen. Die pappel, die mir hilft die nötige Tiefe für die Gespräche mit den anderen Bäumen zu erlangen, schickt mich zur Birke. Ich muss estwas suchen, bis ich im betonufer eine zarte, krumme Birke finde, die sich ihren WEg und Platz erkämpft hat. Sie ist eine der wenigen Birken am Birsköpfli unmittelbar in meiner Nachbarschaf, die nicht von Menschen an ihren Platz gepflanzt wurde. Eine ,wilde' Birke.

Sie macht mich mit der Tagesrune bekannt Hagalaz \textarc{[\withlines]h}. Eine kleine Meise sitzt gemütlich genau vor meienr nase auf der Birke und frisst.

Nachdem mir die Birke die Monatsrune mitgeteilt hat, möchte sie lieber über ,ihre' Rune reden. Sie hat eine Rune, die nach ihr Berkano, Berkana benannt wurde \textarc{[\withlines]b}.

Die Birke bleibt immer jugendlich. Sie ist stark aber biegsam und ihre junge Rinde strahlend weiss. Die Birke ist die Hüterin der Gegensätze und ihre Rune Berkan \textarc{[\withlines]b} zeigt dies, indem sie zwei abgeschlossene Bereiche nebeneinander, gleichgross beinhaltet. Einer oben, einer unten. Ohne Wertung einerseits, aber andererseits ohne Verbindung. Oben und unten stehen gleichberechtigt über einander.

In der Birke selbst sind weitere Gegensätze versteckt. Sie erneuert sich an der Rinde oft und diese trägt den weissen, zarten, dünnen Mantel der Jugend, gleichzeitig ist die alte borkige Rinde, die eine alte Birke mit der Zeit ausbildet, tief gefurcht und sehr dunkel, bis schwarz. 

Das Birkenpech, das ebenso tiefschwarz ist und wird für die Herstellung von Werkzeugen seit mindestens 200.000 Jahren vom Menschen verwendet.

Die Birke macht mich auf etwas aufmerksam, dass ich am Tag zuvor in einem Youtube-Video am Rande aufgeschnappt hatte. Nämlich, dass die Trauerkleidung in den nordischen Ländern vor langer Zeit weiss gewesen ist. Und später gewechselt hat in schwarz und das in den alten Zeiten schwarz die ,positive' Farbe und die Farbe der Materie war und weiss die Farbe für die Geister und das Totenreich und damit die Farbe der ,Trauer'.

Um Birkenpech herzustellen wird die weisse, trockene Rinde durch Feuer zu schwarzem, klebrigen Pech Verwandelt. Um bei nassem Wetter ein Feuer in Gang zu bringen empfiehlt es sich Birkenrinde als Anzünder zu nehmen. 

Die Birke wächst und indem sie sich bewegt, ist sie der verbindende Zwischenbereich der beiden Bereiche von Berkana \textarc{[\withlines]b} \footnote{siehe Abbildung}. Sie ist der kleine Bereich zwischen dem oberen und dem unteren Dreieck der Rune.

Sie kennt beide Bereiche und gibt keinem den Vorzug. Sie weiss, dass sie immer beide braucht. Oft finden wir sie in Zwischenbereichen, so wie am Rheinufer im harten Beton/Stein. Im Moor, dort, wo die Zeit stehen geblieben ist, oder auf Halden, am Rand halt\dots...


\section{Die Birke und die verwandlung von Berkano \textarc{[\withlines]B} zu Tag, Dagaz \textarc{[\withlines]d} }

Die Birke zeigt mir eine weitere Rune Dagaz \textarc{[\withlines]d}.
Sie ist (fast) eine Spiegelung von Berkano. Sie zeigt zwei geschlossene Räume an, die jedoch links und rechts, oder waagerecht angeordnet sind. Berkano um 90 Grad gedreht. Dagaz heisst Tag und es kann nur einen Tag geben, wenn es auch eine Nacht gibt. So steht diese Rune Dagaz für eine anderen, waagerechten Raum von Gegensätzen.

Hier, bei der Birke und ihrer Rune treffe ich auf einem Baum, für den Wasser und seine Symbolik zum Emotionalen sehr wichtig sind.\footnote{Nur damit wir uns richtig verstehen, jeder Baum findet Wasser zum Trinken toll. Nur der emotionale Anteil, den Wasser als Sinnbild trägt, ist nicht für jeden Baum in diesem Gespräch wichtig. Was sich am nächsten Tag vielleicht schon geändert hat!}

Die Birke kann uns Menschen mit ihren gegensätzlichen Aspekten genau in diesem Bereich, dem der Gefühle helfen und uns unterstützen Emotionen zu verstehen und umzusetzen.

die Birke hat nicht nur zum Element Wasser eien intensive beziehung, sondern mit ihren kleinen, gezähnten Blättern und langen Rispenblüten auch zur Luft und durch ihr Rindenpech und Brennbarkeit natürlich auch zum feuer. Die Birke kann uns helfen unser inneres Feuer nutzbar (Birkenpech $\rightarrow$ Werkzeug basteln) machen können.

Ein weiterer Aspekt der Birke für uns Menschen ist ihr Bezug zum positiven, lichten, lieblichen Feenanteil, zur holden, weissen Maid, Jungfrau, etc. Über das Pech ist sie mit Frau Holle, die eher die schwarze Göttin verkörpert, wenn es um dieses geht, verbandelt: Schliesslich wurde die faule Marie mit Pech übergossen.

 Die vielen verschiedenen Bereiche, die die Birke mir zeigt und die vertraute und gut verständliche Art, wie die Birke mit mir spricht, zeigt wie vertraut birke und Mensch seit vielen tausenden Jahren sind/ waren?
 
Heute am strahlend sonnigen Neujahrestag, an dem alle frei haben, ist es sehr voll. Spaziergänger sind überall udn mit ihnen, wie überall, wo viele Menschen sind, viele Gedankenstrukturen. Das Kollektive Bewusstsein, das sich alle Menschen teilen und täglich weiterentwickeln und füttern, ist dann natürlich ebenso grösser, lauter, präsenter.

Für uns Baumsprecher ist es gut sich an ein Training, ein Baum Gespräch zu wagen, wenn viele Menschen unterwegs sind. Aber es ist viel anstrengender, denn es braucht mehr, z.T. sehr viel mehr Aufmerksamkeit. Ich bedanke mich bei der Birke. 

Sie gibt mir auf den Weg, der Baum ist mit allem eins, er wertet nicht. Für hat Berkano \textarc{[\withlines]b} in dem Sinne eine andere Bedeutung, der Baum hat keine Emotionen. Für den Menschen kann es sowohl Berkano\textarc{[\withlines]b} als auch Dagaz \textarc{[\withlines]d} sein.


\section*{Das Eibenklapprunenspiel geht weiter, Mann \textarc{[\withlines]m} }

Heute mach ich auf demRückweg halt unter meinem Eibenfreund. 

Die Eibe spiegelt mir die nächste Rune herbei.

Die erste Spiegelung war waagerecht durch die RUne Eiwaz\textarc{[\withlines]I} und dadurch entstand die Rune Teiwaz, die wie ein Pfeil aussieht \textarc{[\withlines]t}\footnote{siehe Abbildung}.

Indem wir Teiwaz \textarc{[\withlines]t} teilen, umklappen und neu zusammenfügen erhalten wir Mannaz \textarc{[\withlines]m}, die Rune Mensch. 

Runen zeigen Bewegungen  und deren Richtungen an, indem sie sich verändern, sich spiegeln, umklappen, verschieben oder auf den Kopf stellen entstehen neue Runen, die einen neuen Ein-wicklung-spunkt zeigen.

Hier macht die Eibe einen sehr konkreten EInwandt: ,,Wichtig! Nierenwärmer halten Nieren warm! Wenn DU morgen wiederkommst, dann ziehe ihn an. Denn ein Mensch, der vor Kälte Schlottert, kann nicht so gut reden!'' Fürsorgliche Eibe!


\section*{Die Pappelfrau gibt Tips für ein Runenorakel \textarc{[\withlines]orakel} }

Ich tappe über die morgendliche frische Wiese zur Pappel. Ich lehne mich mit dem Rücken an den Stamm und ,klopfe' bei der Pappelfrau. Und schon spüre ich wie hinter mir die Türe auf geht, ich kippe hinein. Die Farben werden klarer, Konturen deutlicher, das Licht milder.

Ich frage, wie ich es mit einem Runenorakel wäre. Die pappelfrau findet, ja, wieso nicht. allerdings findet sie es merkwürdig, wenn für ein das Orakel Holzscheiben genommen werdne. So wie die Bäume mir die Runen zeigen, hat es nichts orakeliges, aber sind finden neue Ideen sehr interessant und benutzten gerne die Dinge, die wir Mensschen in unseren Köpfen mitbringen, um uns ihre Welt, die so anders ist, einigermassen erklären zu können.\footnote{Ich glaub, ich habs schon erwähnt, aber ich kanne s nicht oft genug erwähnen: deshalb sind die Wahrnehmungen der verschidenen Menschen eben so unterschiedlich, bzw. das, was die Bäume ihnen erzählen. In jeden Menschen passt nur ein kleiner Teil der Baumwelt und die Bäume erklären ihm alles so gut wie möglich in seinen Worten. so sind sie, die Guten!}

Das  mit dem Orakel ist für die Pappelfrau interessant, weil die Menschen die Runen dafür benutzen. Sie selbst und die anderen Bäume benutzen sie anders. Ich überlege, was sich denn eignen würde. 

Rheinkiesel? Von denen gibt es genug und in allen möglichen Farben. Die Idee findet die Pappelfrau gut. Die Steine brnigen als Flussszteine Erfahrung mit, Lebenserfahrung und die ist wichtig zum Orakeln. 

Runensteine, also = gut!

In verschiedenen Farben und Formen. Einige Steine mit Quarzbändern haben vielleicht schon das richtige Muster. Wir überlegen: Für jede Rune einen Stein finden, der deren Raum in Struktur, Farbe und Grösse ähnelt? Die Steine sollten handlich, nicht zu gross sein\dots müssten aber nicht alle genau gleich sein.

Farbe? Die Runen farbig darauf machen. Ich sehe sie rot vor mir, Karminrot oder Zinnober\dots mit welchem Färbematerial? Oder Gold?

Die Pappel macht mich auf folgendes aufmerksam, ihrer Meinung nach würde die gelegte Rune ungefähr ein Jahr brauchen um ihren Raum gänzlich zu entfalten! uff! Das ist für uns Menschen eine lange Zeit! Die Pappelfrau meint, also ein Monat wäre schon serh knapp um mit einer RUen zu arbeiten!

Auf keinen Fall aerber kürzer, weil die RUne sich sonst nicht entfalten kann!

Runen sind mit ihrem Bewegungstempo an die Natur gebunden, da sie schlicht Naturräume sind. Sie sind damit an die Kreisläufe der Natur, damit der Planeten, der Jahresläufe etc. angesclossen und wirken dadurch und darin.

$\Rightarrow$ sie sind nicht immer langsam

Auch Plötzliches ist möglich, Bsw. erdrutsch, Gewitter\dots

Inzwischen sind wir usn auf einer anderen Ebene mit der Schrift einig: Rot, am liebsten Zinnoberrot. Ich habe die Idee Siegellack in der Papiermühle zu kaufen. Dieser hat ein sehr kräftiges rot und besteht auch aus Wachsen und Harzen = einbisschen Natur und ein bisschen edler!

Ich frage, ob die RUnen dann irgendwie aufgeladen werden müssen\dots unter einem Baum vergraben o.ä. Nein! Müssen sie nicht!

Und alle RUnen gehören allen Bäumen.

$\neq$ Ein Baum eine Rune für sich

$\rightarrow$ allerdings sind die Bäume jeder einzelne und jede Art/ familie eine Persönlichkiet $\Rightarrow$ diese kann mehr Ähnlichkeit und Nähe zu einer bestimmten Rune haben als eine andere Art, ein anderer einzelner baum (der selben Art oder einer anderen).

Aber alle Bäume benutzen alle Runen

$\rightarrow$ dabei kann dann ein einzelner BAum mehrere Runen schicken:

Eine für den Aspekt, den er in der Sache vertritt

und eine die Möglichkeiten oder einen Bewegungsraum (ent-wicklung) anzeigt

Wichtig! sagt die Pappelfrau: ,,Keine Einschränkungen, weder für die Bäume, noch für die RUnen.

Wichtig! ,,Runen ziehen = wie mit der Natur und da vor allem den Bäumen telefonieren!''

Sie geben der Unterhaltung einen Raum und damit einen Rahmen, eine spezielle Richtung, eine spezielle Telefonleoitung und in diesen Gespächsraum kann dann der Baum eintreten, der für dieses Thema, diese Leitung gerade Experte ist.

$\Rightarrow$ Runen verbinden den Menschen mit einer spezifischen Bewegung der Natur auf die diese dann feinstofflich antwortet.

Die Hugin/Munins haben das Gespräch während ihres Zmorgens auf der Matte aufmerkasm verfolgt. Hin und wieder mischen sie sich mit einem ,,Kraah, Kraah!'' ein. 

$\Rightarrow$ Runen sind wie eine Beschwörungsformel

$\Rightarrow$ Mit Respekt und Sorgfalt behandeln, meldet sich die Rose vom Birsufer\footnote{Ja, da sich die Bäume alle auf der Ebene befinden, wo alles miteinander verbunden ist, können die interessierten ,mithören' und sich ,einmischen', wenn sie etwas ergänzen wollen! Sie wissen immer alle, alles!\dots aber keine Sorge, sie sind sehr diskret und haben auch meistens anderes zu tun\dots}

$\Rightarrow$ Runen ziehen = wie Ritual: rune wird gezogen (Bsp. aus einem Fellbeutel/ speziellen, weil handgenmachten o.ä. Stoffbeutel) indem sie ertastet wird.

Dann kann sich der Mensch in die Rune vertiefen, bzw. die kräfte der RUne, die bewegung der Rune, deren Raum rufen $\rightarrow$ spüren

dazu ist weihrauch, Beisfuss, Salbei, u.ä. hilfreich $\rightarrow$ Rauch = Verbindung von Luft und Feuer = es ist Kommunikation (Verstofflichen eines Gespräches, eines Momentes, einer Verbindung, die sich wiede löst)

Dazu gehören aber auch Erde und Wasser $\Rightarrow$ daher die RUne auf den Rheinkiesel malen $\Rightarrow$ Runenstein

Runenstein zum Spüren in das Wasser legen

in eine shcöne Schüssel,

an ein seichtes Ufer (See/ Fluss)

Hugin/Munin kräch\dots Pardon, singt die ganze Zeit! 

$\Rightarrow$ Spüren: Wie genau das Ritual beginnen und wie genau beenden! Worte, etc.

Uff! Das ist ganz schön schwierig. Die Bäume wundern sich etwas, aber sie helfen mit, so gut sie können. 

Die Pappelfrau versucht mir schmackhaft zu machen mit den Runen Dinge zu fühlen. Und mit den RUnen wie mit einem Inenren Werkzeugkasten zu arbeiten, wie die Bäume es in unserem Kulturkreis machen.\footnote{Und vor Urzeiten Bäume und Menschen zusammen}


\section*{Der Ahorn über Menschen \textarc{[\withlines]m} }

Ich stapfe nachdenklich über die Wiese zu dem Bergahorn. Das Gespräch mit der Pappelfrau hat nicht einmal 10 Min. gedauert. Ich bin immer wieder erstaunt wie viel Informationen in kürzester Zeit hin und her gespürt werden können, für die Menschen bei einem Gespräch mit Worten und linker Gehirnhälfte mindestens eine Stunde oder länger bräuchten\dots

Der Bergahorn lädt mich ein seine Version des Menschen kennen zu lernen. 

In der Natur ist der Mensch mit allem als einzigartiges Wesen, einzigartige Bewegung verbunden.\footnote{Also auch mit den anderen Menschen, die ja auch Natur sind!}

Wenn Menschen mit Mneschen zusammen sind, dann sind sie wie alle Menschen, wie

die Samenkörner des Ahorns:

$\rightarrow$ Alle Menschen haben ein Fünkchen Geist/Feuer + Flügel

$\rightarrow$ Alle sind gleich ausgestattet

$\Rightarrow$ Jeder Mensch, absolut jeder Mensch = ganz Natur

er trägt alles + viel mehr in sich

Jeder Mensch = ein ganzes Universum

$\Rightarrow$ anders ausgedrückt: Jeder trägt eine ganze Welt in sich, unendlich gross

Und doch

ist jeder Mensch unter Menschen gleich, wie ein Samenkorn des Bergahorns\footnote{siehe Abbildung}

Und dennoch gibt es Unterschiede: einige trauen sich los zu lassen und fliegen

und davon haben einige Glück od./u. dei Kraft und wachsen dann an einem anderen, neuen Ort zu einem Bergahorn

Einige bleiben am Baum,

sie sind nicht besser/schlechter/fauler als die anderen, die nicht am Baum blieben\dots denn diese haben einfach die Aufgabe am Baum zu bleiebn und bsw. mit den anderen Samen, die am Baum hängen im WInter ein wunderschönes Lied zusammen im Chor zu rascheln/flüstern

Vielleicht fehlte ihnen zur rechten Zeit der Wind und so blieben sie.

Jedes und alles ist an seinem rechten Ort.

Es war ein sehr freundliches und liebevolles Gespräch, denn der Bergahorn sorgt sich wohl um das, was er von den Menschen spürt, die an ihm vorbei spazieren/joggen/rennen, Hunde ausführen\dots  


\section*{Frau Hasel, die Ahnenfrau schenkt Uruz \textarc{[\withlines]u} }

Die Haselfrau: Gute Mutter

Schutz, weich, wie ihre flauschigen, pelzigen Blätter, fruchtbar (Sie hat jetzt Anfang Januar rote Blüten und lange gelbe-grüne Kätzchen.)

Sie gehört zur Birkenfamilie.

Sie ist ein Busch, sie hat viele Stämme (Auch fruchtbar, biegsam, schmiegsam

und laut Wiki ist sie so jung geblieben, dass sie nur Rinde und keine Borke bekommt.

Das Holz von Frau Hasel ist aber durchaus als Bogen treffsicher und wird gerne für den Bogenbau verwendet.

Die Nuss bietet für Frau Hasels Kinder Haus und Schutz
= ein Aspekt der Allmutter Holle, den die Haselfrau repräsentiert

Die Holle-Göttin selbst ist viel grösser und es braucht mehrere Bäume, um all ihre Aspekte zu zeigen.

Frau Hasel ist die gute Mutter, die liebevolle, fördernde, nährende Mutter und eine der roten Göttinnen

$\rightarrow$ während der Holunderbusch zum Reich der schwarzen Göttin gehört und
$\rightarrow$  die Birke zu Reich der weissen Jungfrau-Göttin

Frau Hasel teilt sich in ihrer Fruchtbarkeit mit und verteilt sich. Ihre Stärke liegt in der Sanftheit, die der ihrer kräftigen, satten, saftig grünen, pelzigen, dicken und z.T recht grossen Blättern ähnelt.

Härte/Strenge? Nein, doch es braucht Fleiss um ihre Früchte zu ernten und zu öffnen (Holle $\rightarrow$ Geld $\rightarrow$ Pechmarie) 

Frau Hasel schenkt ihren Liebsten, ihren Kindern eine Schutzkapsel und damit grenzt sie ihr Inneres, ihre Fülle ab. Was nicht heisst,d ass sie die fleissigen nicht gerne ernährt, nur\dots sammeln und knacken muss er sie können, die Schatztruhe von Frau hasels Reichtum

Ihre Fruchtbarkeit lebt Frau Hasel völlig ungeniert Sie hat eine menge Pollenstaub zu vergeben udn trägt selbst zahlreiche kleien rote Blüten, die direkt aus den feinen Ästchen wachsen. Männliche und weibliche Blüten befinden sich auf einem Baum, auch wenn die haselbüsche sich gerne austauschen mit ihren Pollen.

$\Rightarrow$ Frau Holle hat ,ihre'Welt stehts beisamen. Und mit ihren zahlreichen Stämmen ist sie gerne mal ein kleiner Wald für sich\dots

$\Rightarrow$ Haus, Familie, Ahnen im positivsten Sinne, sind frau Hasels Anliegen

$\rightarrow$ in ihrem Wurzelbereich liegt dann auch der Eingang in die Welt der unterstützenden, schützenden Ahnen:

3 Nüsse für Aschenbrödel!

Doch wie die Pech- und die Goldmarie bestätigen können mit den Gaben, die aus Frau Holles Reich kommen, solltest Du immer respektvoll umgehen 

Familie $\leftrightarrow$ Samen

eine Beziehung, die in beide Richtungen funktioniert

Familie, gute Ahnen (im Sinne von wohlgesonnen)

Sie sind Vertraute $\neq$ Freunde

Für das Jahr gibt sie mir die Rune Uruz, Auerochse \textarc{[\withlines]u}

Frau Hasels Liebling ist Berkano \textarc{[\withlines]b}, liegt in der Familie. Die Hasel macht auf das untere Dreieck vonBerkano aufmerksam, das sie als den dicken Bauch einer Schwangeren betrachtet.

\section*{Der arme Ritter Hain bringt die Gabe, Gebo \textarc{[\withlines]g} }

Neben Frua Hasel steht der RItter Hain. Dem Baum geht es nicht gut, er mag nicht viel reden. Er hat im Jahr 2018 bei der grossen Trockenheit sehr gelitten und sich bis jetzt nicht ganz erholt. Er hat viele Vertrocknete Blätter an den Ästen, die er nicht abwerfen konnte, weil sie vorzeitig an seinen Ästen vertrocknet waren.

Aber, der tapfere Ritter gibt mir eine Rune: Gebo, die Gabe \textarc{[\withlines]g}

Da ich von dem Baum selbst nicht viel erfahren kann, schaue ich in der Wikipedia nach.\footnote{Das ist mein oft beschrittener Weg, wenn ich eine Pflanze oder ein Tier genauer oder neu kennen lernte: Ich lese mal, was Wiki sagt. Dabei relativieren sich dann die Dinge und das ist nützlich. Manchmal verstehst Du Informationen falsch, weil Du schon eine Vorstellung von dem Wesen hast, oder bestimmte Zusammenhänge nicht kennst und Dein Gehirn dann Dinge ergänzt, die es nicht ergänzen sollte. Das ist normal, das Dein Gehirn Dinge ergänzt, aber für den Umgang mit der feinstofflichen Welt ist es wichtig, dass Du diesen Vorgang berücksichtigst und bereit bist, diese Ergänzungen, wenn Du bemerkst, dass sie falsch sind, wieder korregierst! Manche haben Angst ihre Gedanken zu korregieren, weil sie befürchten dann dumm da zu stehen. Aber das ist unnötig! Und Menschen, die mit dem Feinstofflichen arbeiten und sagen, dass sie die Informationen ,,nie'' falsch verstehen\dots ist schwierig! Denn in einer Kommunikation kann man immer mal Fehler machen! Das ist normal.

Deshalb ist es gut eine möglichst sachliche Informationsquelle als Sachgrundlage zu befragen. Und! Nicht vergessen! Wiki ist ebenfalls von Menschen gemacht, d.h. es ist begrenzt. Wenn Du also Dinge erfahren hast, die nicht im Wiki auftauchen, dann müssen sie nicht falsch sein, sondern Du musst Dich nur weiter daran machen sie in anderen, seriösen Quellen zu erforschen! Das wichtigste ist, die Informationen nicht in feste Schubladen im Gehirn zu verstauen und nie wieder herauszuholen oder anzuschauen. Bleib flexibel und forsche stetig!}

In der Wikipedia erfahre ich wie hart das Holz der Hain- oder weissbuche ist! es ist härterr als das von Buche oder sogar härter als Eiche. Und es ist, der Name verrät es schon: Weiss.\footnote{https://de.wikipedia.org/wiki/Hainbuche}

In unserem Reich von König Buche und Königinmutter Hasel ist der Hain der jugendliche, tapfere, Weisse Ritter und jugendliche Liebhaber, der schwärmerische Minnesänger. Die Buche ruft: ,,Er ist Vasall!''

\section*{Kirsche, die schöne Königin und die Not, Naudiz \textarc{[\withlines]n} }

Ich lerne eine neue Schönheit aus der Rosenfamilie kennen: Die Kirsche. Ihr Reich ist die Schönheit, weibliche Schönheit, die selbstbestimmte Weiblichkeit.

Auch sie ist, wie die Hasel mit der roten Göttin verbandelt, aber unter einem anderen Aspekt von ihr. 

Die Kirsche kann über den Aspekt der Herrscherin lehren, der selbstbestimmten und bestimmenden, herr-schenden Frau

$\neq$ Stacheln, wie die Rose

Sie ist die Königin der $\heartsuit$

Das ganz wörtlich, denn jedem schenkt sie im Frühling nicht nur ihre zarten, weiss-rosa Blüten, sondern etwas später ein kleines süsses, rotes, saftiges Herz, die Kirsche!

= Die Luftkönigin: Luft + Feuer

Die Kirschen tragen das Feurige der Samen auch Aussen.

Der Kern wird von der Kirsche, ihrem Fruchtfleisch einghült und darin von den Beschenkten Menschen, Tiere besonders Vögeln mitgenommen. Der Same selbst jedoch ist in einen sehr harten Kern gehüllt. Er enthält Amygdalin, das auch in Bittermandeln vorkommt und Blausäure enthält. Er ist also giftig, da er stark mit dem feinstofflichen verbunden ist.

$\Rightarrow$ Feuer/ Leidenschaft ist in der Frucht sichtbar

(Pflaumen, Aprikosen und andere Rosaceae-Angehörige sind fruchtbarer $\neq$ so luftig,elegant, kommunikativ und nicht so lustvoll wie die Kirsche)

Die Kirschenkönigin beherrsct die lustvolle Kommunkation perfekt. Sie ist in ihrer Erscheinung mit der metallisch glänzenden, längs gemusterten Rinde elegant und vornehm an Wuchs.

In der Konversation gibt sie jedem ein Herzchen, aber nicht ihr Herz!

Die Kirsche ist eine gute Schönheitscoachin: Sie lenkt meinen Blick auf ihr eschlanke Gestalt und ihr eschimmernde Rinde: Dabei mit sie nicht schlank im Sinne von Diät, sondern einen schönen wohlproportionierten Körper, der innerlich und äusserlich beweglich ist ($\neq$ Matrone, auch in der Kleidung)

Kirsche liebt die Vögel als Boten der Luft = Freunde im Gesite, der Luftigkeit, Leichtigkeit. Sie werden reichlich mit den Kirschen beschenkt und leihen der Königin auf diese Weise ihre Flügel, indem sie ihre Samen verteilen.

Die Blätter der Kirsche sind gross, länglich-oval und gezähnt. Sie lässt sie gerne im Wind flattern und rascheln.

Sie sind ihre Flügel mit denen sie die Luft spürt, fächelt und die Sonne einfängt. Die Blattränder sind feurig gezähnt.

Die Kirschkönigin gibt mir ihre Jahresrune: Naudiz, Not \textarc{[\withlines]n}.

Ich mache wohl recht grosse Augen. Die Königin sagt zu der Ruen: ,,In der Not hilft es Haltung zu bewahren und tapfer Herzchen, aber nicht das Herz $\heartsuit$ zu verschenken.


\section*{Kirsches Wonne, Wunjo \textarc{[\withlines]w} }

Kirsches Lieblingsrune: Wunjo, Wonne \textarc{[\withlines]w}

,,Wie Du bemerkt hast, ohne Bauch!'' sagt die Schöne.

,,Esse viele Kirschen, denn sie machen schön. Und einen schönen roten Mund! Und Innerlich machen die Kirschen leicht! Und Das führt zu eleganter Sexualität und Leidenschaft! Und nicht gleich zu einem Stall voll Kinder!'' \footnote{Wer hätte gedacht, dass Bäume Sexberatung bieten können!}

,,Es gibt verschiedene Arten die Fruchtbarkeit auszuleben! Meine  Kirschen sind für mich kleine leidenschaftliche ,Geistesblitze'!'' sagt sie.

Der Raum der Königin, wie sollte es anders sein. Das Boudoir des Shclosses. Mit Hofdamen, Kammerzofen und jugendlichen Liebhabern\dots oder

der üppige, lichte, leichte Barockpark mit vielen Vögeln, Schwalben

Sie hat eine Schwäche für die Künste und Künstler!

Ich verabschiede mich von Ihrer Majestät mit einem kirschmundkuss derselben. Und mir wird bewusst, dass sich die Kirsch mitd en anderen Obstbäumen verglichen hat$\Rightarrow$ Sie hat ein Ego!

\section*{Birken Wellness \textarc{[\withlines]} }

Die Birke steht gleich einige meter weiter am Rheinboard. Sie hat Gesellschaft von einer Meise und einem Kormoran. Die Meise ist ein munteres Plappermäulchen. Der Kormoran dagegen hat die Fähigkeit himmelhoch zu fliegen und tief ins Dunkle (Emotion) einzutauchen

Die Birke zeigt mir noch einmal ihre Gesetzmässigkeiten der Gegensätze:

Weiss $\rightarrow$ $\leftrightarrow$  $\leftrightarrow$ $\leftarrow$ Schwarz


Alt $\rightarrow$ $\leftrightarrow$  $\leftrightarrow$ $\leftarrow$ Jung

,,Schau, das Geheimnis der Birke ist, diese Widersprüche stehen lassen können! Das ist ein Gehirntraining, denn Euer Gehirn mag es, wenn alles zusammenpasst und schön glattgebügelt ist. Aber das Gehirn ist vile zu klein, um all die Zusammenhänge zu begreifen, deshalb ist es gut, wenn Du Dein Gehirn darin trainierst Gegensätze und Widersprüche stehen lassen zu können.''

Die Birke kann dabei helfen! ,,Rede also mit Deiner Birke!'' 

In der Pflanzenheilkunde wird die birke zur ,,Verjüngung'' eingesetzt.

,,Ich durchlichte und durchlüfte die DInge und helfe die schweren DInge loszuwerden.'' (Astringierende Wirkung)

,,Ein jugendliches Gehirn kann Widersprüche besser stehen lassen, weil es neugieriger ist und noch niocht so viele festgefahrene Glaubenstrukturen hat.''


,,Also: Im Frühling hilft tee aus Birkenblättern Altes, Glattgebügeltes loszulassen und sich fit zu machen für neue Widersprüche.''

,,Denke doch an die Schwitzhütten, oder an die Saunen im NOrden, dort schlägt man sich gegenseitig mit Birkenreisig auf den Rücken, das fördert die Durchblutung, sagt man und feinstofflich und grobstofflich transportiert es die altenBrocken ab und reinigt!''

Birkenholz hält jung und beweglich.


\section*{Die Welt in der Rune \textarc{[\withlines]} }

\section*{Die Kiefer als Baumpriester und Jera, die Jahr-Rune \textarc{[\withlines]j} und Kaunan, das  \textarc{[\withlines]k} }

Die Kiefer hinter meinem Haus, die mit ihren Brüdern im gleichen Garten wohnt wie König Buche, meldet sich, weil sie möchte ein interview. Und da höre ich meinen eigenen, jungen Baum rufen, eine kleine Esche, sie möchte ebenso etwas beitragen.

Die Kiefer beginnt. Sie ist sehr luftig. Ihre Nadeln sind dünne, lange Fühler für Luft und Licht. So ist sie dem Fuer stark zugewandt. Dies sind alle Pflanzen und Bäume, die stark Harze und atherische Öle herstellen.

Mit den Nadel, die aus festem Materieal bestehen, verliert die Kiefer nicht viel Wasser durch Verdunstung, sagt sie. Sie schwitzt nicht.

,,Öle sind feinstofflich durchlichtete Substanz. Sie entstehen, wenn die Pflanze das Sonnenlicht/ das intensivste Feuer in ihrem Inneren sich zu eigen macht und mit ihrer spezifischen Bewegung prägt und verwandelt.''

Uuih, das ist kompliziert. Kompliziert für mich in Worte zu fassen, denn wir Menschen produzieren in uns keine Öl\footnote{Nope, Fett ist kein Öl!}. Daher kennen wir die Begriffe und Bilder nicht, die für die Ölproduktion sinn-voll wären.

Der Raum der Kiefer ist hell, licht, geistig/feinstofflich stark präsent wie ein sakraler Raum/ eine sakrale Halle, eine Kirche, z.B.

Die Buche ruft dazwischen.\footnote{1-2 Tage redeten alle Bäume in einem riesigen Gesprächsplenum fast pasuenlos mit mir. Sie riefen dann in das aktuelle Gespräch hinein, um ihrerseits wichtiges zu dem Thema losuzuwerden. das war mega, mega spannend! Und anstrengend, dann machte ich so gut es ging Pause.}

Die Buche ruft dazwischen: Sie zeigt mir einen Buchenwald. Die Buchen, lauter Könige, die sich um die wette nach dem Sonnenlicht reckten und streckten und schliesslich unter ihren Kronen eine wunderschöne, elegante und oft sehr hohe Säulenhalle s
entstehen liessen. Diese Säulenhalle, die wir heute nur von Kirschen kennen, war früher auch ein Symbol für die Königshalle.

Diese alte Form von Königtum, bei der mehrere Könige/Adlige einen gemeinsamen Raum bilden, konnte für die Untertanen ein Schutz sein. 

,,Ja, also nun\dots '' die Kiefer meldet sich zurück

Kiefern sind die Priester unter den Bäumen. Sie hüllen sich und ihre Umgebung mit ihren ätherischen Ölen und Harzen in eine meditative Stimmung. (Vor allem, wenn die Sonnenstrahlen die Öle wärmen und aktivieren.) 

,,Mit unseren vielen, langen Nadeln sind wir in einem sehr regen, intensiven Austausch mit allem um uns herum! Hier kannst Du spüren, wie stark wir mit dem Element Luft (Kommunikation/ Raum) verbunden sind. Und das auf eine feurige, also feinstoffliche Art und Weise.

Unsere Zapfen sehen aus wie kleine Pagoden! Sie sind kleine Tempel, die in einer Spiralform aufgebaut sind. Alle Zapfen sind so, aber unsere Kieferzapfen und die Pinienzapfen am meisten.

Die Runen? Ja, die helfen feinstoffliche Dinge aufnehmen und die spirituelle Ordnung der Dinge zu verstehen.''

Die Kiefer gibt mir die Rune Jera, Jahr \textarc{[\withlines]j}.

,,Jera \textarc{[\withlines]j} passt sehr schön zu unseren Zapfen!

Während Ingwaz, die Feuerrune  \textarc{[\withlines]\ng} gut zu unseren Nadeln passt. 

Und Isa, Eis \textarc{[\withlines]i} gibt uns in unserem Stamm die Konzentration, die wir für unsere Meditation brauchen.

Das sind die wesentlichen Aspekte vom Raum Kiefer oberhalb der Erde. Unser Kopfbereich, die Wurzeln kennen viele Details, die uns über unsere Nadeln, unsere Sinnesorgane zukommen.

Wir Kiefer haben eine recht starke ,Aufbautrennung'. Unser Stamm, unsere Äste und unsere Nadeln sind verschiedene Räume.\footnote{Siehe Abbildung}

Unsere Samen können viele Tiere ernähren. Und viele Samen teilen sich einen Zapfen, eine Pagode, einen Raum. Wie es sich für einen Tempel gehört.''

Die Kiefer liebt die japanischen und chinesischen Tempel. Sie sind wie ,,Zapfen'' für Menschen!

Der Buddhismus ist ein Bild für diese ,,Zapfenkultur''. Das feinstoffliche geht bis in das körperlich-grobstoffliche mit Übungen hineingetragen. Und die Pagode, oder der Tempel ist der Austauschort dafür:

$$ feinstofflich \leftrightarrow grobstofflich $$

$\Rightarrow $ Durch die körperlichen Übungen, verbinden sich feinstofflicher und grobstofflicher Körper gut miteinander und dadurch kann das Ego auf ein Minimum reduziert werden.

\section*{Guter Freund Eschi, das Kreuz und Raido \textarc{[\withlines]r} und Sowulo \textarc{[\withlines]s}}

Meine kleine esche, Eschi, möchte auch interviewt werden. Und obwohl er bei uns auf dem Balkon und Fensterbrett aus dem Samen geschlüpft ist und ein Baumbaby von 8 Jahren, kann er einiges zu den Rauhnachtgesprächen beitragen. 

,,Wir Eschen spielen im Nordreich, im Reich der Runen eine bedeutenden Rolle. Mag die Weltenesche immergrün sein und eigentlich eine Eibe, so ist die Geschichte der Menschen und der Eschen eng verknüpft.

So heisst es in der Edda, die ersten Mneschen Askmund Emlba wären aus Holz entstanden. Ask, der Mann, könnte aus Eschenholz entstanden sein.

Womit wir Eschen seit unserer Entstehung besonders verbunden sind, ist das Licht. Jeder Baum braucht Licht, ist ja klar. Aber wir Eschen können das Licht auf eschige Art und WEise speichern und das können nur wir! Deshalb haben wir im Herbst die leuchtend gelben Blätter, wenn wir gesund sind. Wir leuchten mit ihnen, ihrem lichten, hellen Gelb die Sonnenstrahlen des Sommers zurück. Unser Laub ist dadurch besonders beliebt bei Pflanzenfressern, denn sie können über uns vom Licht essen.''

Dann kommt Eschi auf ein Thema zu sprechen, das bei uns Menschen s hwierig ist, weil es sich seit seinen Ursprüngen vor ca. 2020 sehr mit Institutionen und vielem anderen gemischt hat.

,,Wir Eschen sind Götterbaume. Und wir sind eng verbunden mit der Christuskraft.''

,,Was genau meinst Du damit!'' frage ich den kleinen Baum. ich kann mir nicht vorstellen, dass er eine Kirche, oder Glaubensstruktur meint.

,,Weisst Du das denn nicht?'' fragt der baum erstaunt. ,,Doch!'' gebe ich zu. ,,Aber wir müssen es erklären, damit möglichst viele menschen verstehen von welcher energie Du sprichst, '' sage ich.

Eschi ist, wie alle Bäume verwundert. ,,Die Christuskraft ist durch das Christuswesen und den menschen jesus von nazareth in den feinstofflichen Bereich der Erde hineingelangt. Und das Christuswesen ist eine spezielle Form, Bewegung der Sonne. Früher hätten die menschen stark vereinfacht gesagt, der Christus ist ein Sonnengott. Das einmalige an ihm ist, dass er sich tatsächlich mit einem grobstofflichen Menschenkörper verbunden hat und so den Tod erleben konnte. AUf diese Weise hat dieses Wesen eine intensive Bindung an die Erde sowohl feinstofflich, ale auch grobstofflich. Der Christus hat die grobstofflichen Körper damals wieder stärker an das feinstoffliche angebunden und dadurch lichter, leichter gemacht.''

Eschi ist ratlos. Ich finde, so ist es gut erklärt. Es gibt genug Quellen, für die, die sich mit diesem thema beschäftigen wollen.

Das Christuswesen ist, soweit ich es bisher spüren konnte, allen Pflanzen sehr gut bekannt. Eben, weil es eng mit der Sonne verbunden ist. Alle Pflanzen reagieren sehr freundlich und kommunikativ, wenn ich ihr Reich mit einem Gruss vom Christuswesen betrete und damit anzeige, das ich es kenne und respektiere.

Wenn ich z.B. einen Wald betrete und in dieser Form begrüsse, dann öffnet sich die Tür zur feinstofflichen Ebene auf sehr liebevolle und fröhlcie, be-geist-erte Art.

Wenn ich sagen sollte,was der Christus ist, dann könnte ich sagen, eine starke, liebevolle und leidenschaftliche Bewegung, die jeden feinstofflichen Körper in schwingung bringen kann. Nicht weniger, vermutlich mehr, aber sicher nichts, was mit den Institutionen oder jeglichen Glaubenstrukturen zu tun hat.

Für mich ist es einen Tatsache, die für sich selbst spricht, sich selbst bewegt und daher auf jede Glauben-Struktur\footnote{Im Sinne von ,Nicht-Wissen'} verzichten kann.

Die gefiederten Blätter der Esche fangen aus der Luft die Geister/Engel, sagt der Baum. Sie haben einen direkten ,,Draht'' nach oben, unten, rechts oder links\footnote{Dort, wo das geistige steckt}. Aus diesem Grund sieht die Rinde, die recht glatt, aber leicht längsrissig ist, aus wie Asche. Die Asche ist der Rest, der von der Vergeistung, vom Feuer übrig bleibt. Verbrennen ist eine Transformation von grob- zu feinstofflich.

,,Nun kommt ein Geheimnis!'' sagt Eschi. ,,Ist es Dir aufgefallen?'' Äähm, nein, muss ich zugeben. Ich bin eben keine Runen-Fachfrau. ,,Die Sonnenrune, Sowulo \textarc{[\withlines]s}, ist ein altes Sonnenzeichen!''

Ich brauche einen Moment, bis ich verstehe, worauf der Baum hinaus will. ,,Du meinst, es ist ein Zeichen wie die Sonne vor dem Christus gespürt wurde?'' ,,Genau!''\footnote{Ich weiss nicht, was das genau bedeutet. Das ist ein eigenes, umfangreiches Forschungsfeld. Die Bäume finden, es wäre an der Zeit eine Rune mit einem Sonnenrad zu benutzen, aber da dieses Zeichen für die Menschen sehr zwiespältig ist, muss es noch weiter untersucht werden.}

Eschi zeigt mir Runen, die ebenfals kreuzungspunkte besitzen: \textarc{[\withlines]g}\textarc{[\withlines]m}\textarc{[\withlines]d}. Bei ihnen öffnet sich die Kreuzung nach oben/unten und zu den Seiten. Diese Form der Kreuzung symbolisiert eine andere Bewegung.

eschen haben schwarze Blattknospen, die das Licht das Licht stark aufnehmen. 

Nun fehlt nur  noch eine Rune: Raido, Reiten\textarc{[\withlines]r}
Diese Rune ist für Bäume schwierig zu beschreiben, denn sie reiten sehr selten\dots

\section*{Die Pappelfrau über die Kundalinischlange und Raido \textarc{[\withlines]r} und Sowulo \textarc{[\withlines]s} die Menschenrunen }

Ich begrüsse an diesem morgen Frau Pappel und bin wieder erstaunt, wie schnell die Bäume Informationen, die ich nur einige Stunden vorher bekommen habe, sofort spüren und in die Gespräche einfliessen lassen.

Ein Freund und ich hatten in der Nacht davor über Chakren gesprochen. Da mein Freund diese wahnehmen kann, hatte ich ihn gebeten diese bei mir anzuschauen.

Zwei von ihnen waren seiner Meinung nach etwas nach vorne gerutscht, was nicht so gut ist.

Die Pappel zeigte mir sehr munter meine Charkren und gab mir Rat. Sie zeigte sie, wie bereits jedem klar sein sollte, genau so, wie der freund und ich sie gespürt hatten. In diesem Fall wie etwa apfelgrosse bunte Bälle, die sich entlang der Wirbelsäule wie Perlen an einer Schnurr aufgereiht hatten.

Dann zeigte mir Frau Pappel, was passiert, wenn die Kundalini-Schlange erwacht: Die Schlange bewegt sich im Wurzelchakra und schlängelt sich dann hinauf, dabei beginnen die Chakrabälle leicht und luftig hin und her zu schwingen, aber

$\neq$ auf und ab oder vor und zurück

Wenn die Bälle verrutschen und sich nach vorne verschieben, weil man z.B. sehr viel damit arbeitet und diese dabei ausversehen nach vorne schiebt, um ,,dichter'' dran zu sein, dann kommt es zu Timingschwierigkeiten, sagt Frau Pappel.

Denn die kleine, aber feine Distanz, die dann für eine minimale Verzögerung oder Beschleunigung der Aktivierung des Chakraballes sorgt, die reicht aus, um Schwierigkeiten zu bekommen.

ich habe oft das Problem, dass ich viel zu viel rede, vorallem, wenn cih erschöpft bin. Dabei bin ich danach oft von mir genervt, weil das Reden zusätzlich anstrengt und ich über Dinge rede, über die ich nicht unbedingt reden wollte. Oder ich rege mich fürchterlich auf und bin gleichzeitig gestresst von dem eigenen geplapper.

rau Pappel meint, ich könnte in solchen Fällen probieren zu spüren, wo sich der orange Sakralchakraball befindet und wenn er vorne in den Bauch gerutscht ist kann ich probieren ihn wieder vor die Wirbelsäule in die Perlenkette zu schieben.

Auch den grünen Herzball könnte ich hin und wieder auf seinen,,Sitz'' überprüfen und ggf. in die Perlenschnurr einreihen.\footnote{Aha! Ich schaute dann unter Chakren nach: Ja, das ist eine riesige Kiste! Was mir am wichtigsten düngte, war, mit den Bildern und den Gefühlen, die Brunos Worte in meinem Inneren Garten ausgelöst hatten, sofort aktiv und NÜTZLICH zu arbeiten. Und ich staunte wieder, wie präsent und aufmerksam die Bäume sind. Ich glaube nicht, dass ich genau sagen könnte, was meine Pappelfreundin über Nacht alles Neues erfahren hat, wenn ich mit ihr am nächsten Tag wieder spreche!}



\section*{Das Finale: Bäume, Runen und ein Mensch \textarc{[\withlines]} }

Neujahrsnacht: Runen sind Räume

sie beschreiben Räume

Und Odin lauschte in neun Tagen und Nächten den Bäumen die Runen ab.



\chapter{Bitte der Eibe: Esse keine Giftpflanzen!}

Eine Bitte an diejenigen unter uns, die glauben, dass sie den Weg in die feinstoffliche Welt abkürzen können, indem sie giftige Pflanzen essen, oder Rauchen oder in irgendeiner anderen Form in ihren materiellen Körper hineinfüllen!

Macht das nicht! Giftpflanzen sind mächtige Wesen! Sie sind nicht giftig, weil sie böse sind, oder weil sie Schaden wollen, oder damit Du Dir mit ihrem Zutun den Körper vergiftest und dafür einen kurzen Einblick in die Anderswelt erhascht.

Giftpflanzen, die für uns Menschen giftig sind, sind giftig, weil sie für den grobstofflichen Teil des Menschen, die Materie im feinstofflichen zu stark sind!

Nochmal gaaaanz langsam zum mitschreiben: \begin{Large}
Giftpflanzen sind giftig, weil sie für den grobstofflichen Teil des Menschen, also für das Hineintun in den (materiellen) Körper nicht geeignet sind! Sie wirken am Besten und hilfreichsten im feinstofflichen Körper!
\end{Large}

Die meisten giftigen Pflanzen wollen Dir nicht schaden, aber sie können eben nicht davon laufen, wenn Du kommst und sie pflückst und futterst!

Aber genau wegen ihrer Beschaffenheit sind sie so mächtig, dass Du sie nicht essen müsstest, um mit ihrer Hilfe die Anderswelt zu besuchen. Sie können Dir starke Beschützer und Verbündete sein, gerade wenn Du sie achtest und respektierst und sie nicht zwingst Dir auf der grobstofflichen Ebene zu schaden.

Natürlich wirst Du, wenn Du die Pflanze zu Dir nimmst auch in die Anderswelt gelangen. Aber Du benutzt den falschen Eingang! Dir werden die wirk-lichen Dinge versperrt bleiben, allein schon, weil Dein Körper mit den Substanzen der Pflanze kämpfen muss. Und weil Du in diesem Zustand weder Deine Aufmerksamkeit halten kannst, noch bewusst in der Verbindung mit der Pflanze bist! Und Du weder Dich, noch die Pflanze, noch die feinstoffliche Welt auseinanderhalten kannst, weil Du keine Ahnung hast, was, was ist! Du verwirrst Dich lediglich und verlierst Deine innere Verbindung zu Dir, weil Du glaubst, DU musst eine Pflanze, Pille, etc. zu Dir nehmen, um in Dich hineinzugelangen. 

Kleiner Hinweis, Du bist schon da! Du bist schon vollständig sowohl feinstofflich, als auch grobstofflich. Du könntest einfach damit beginnen Dich selbst einmal wahr-zunehmen, bevor Du diese Aufgabe anderen übergibst (Z.B. einer Pflanze).

Natürlich wirst Du die Einheit spüren, aber die ist auch ohne Substanzen nur ein Augenaufschlag von Dir entfernt und es ist ein dummer Trugschluss, wenn Du meinst, nur über eine Substanz und nur für deren Wirkzeit mit allem Verbunden zu sein.

Natürlich haben bestimmte Menschen schon seit Jahrtausenden diese Pflanzen benutzt und z.T. auch zu sich genommen. Dies geschah, weil sich die Menschen von der Einheit entfernten. Das ist so! Das ist der Weg! Aber diejenigen, die die Pflanzen assen, waren darin ausgebildet!

Vergiss nicht, dass die Menschen, die diese Pflanzen kulturell einsetzen, ausgebildet worden sind. Dabei ist es gleich, ob sie eine schamanische Ausbildung, oder eine medizinische, oder eine theologische Ausbildung bekamen und von wem. Diese Menschen sind ausgebildet und haben dadurch auf mindestens einer Ebene und über mehrere Jahre eine enge Verbindung zu der Pflanze aufgebaut.

Wenn Du diese Ausbildung in die Geheimnissen der Pflanze nicht erfahren hast, dann nehme sie nicht zu Dir und zwinge sie nicht, Dir zu schaden!

So bittet die Eibe! (Und ich auch!)\footnote{Sieh' es einmal so, eine Eibe dazu zu zwingen, Dich zu vergiften, ist ungefähr so, als würdest Du einen Elefanten zwingen sich auf Dich drauf zu setzen\dots Und dann würdest Du danach sagen: ,,Oh, ich habe den Elefanten gespürt\dots '' man kann sich auch mit dem Holzhammer auf den Kopf hauen\dots}




\chapter*{Anhang}


\section*{Modellarbeit }
Wichtig, wichtig, wichtig! Für uns Menschen ist es einfacher mit einem Model zu arbeiten! Aber, das ist es nur, wenn wir die Regeln der Modellarbeit beachten! Das ist sehr wichtig, denn, wenn wir vergessen, dass wir esmit einer Gedankenstruktur, einer Vor-stellung zu tun haben, dann kann es schnell passieren, dass wir sie mit der Wirk-lichkeit verwechseln.

Das ist schlecht! Denn die Wirk-lichkeit ist kein Modell, sie enstetht jeden Augenaufschlag komplett neu aus dem AUgenaufschlag davor und aus der Absicht von allen Wesen, die sich mit unserem Universeum beschäftigen. Wir Menschen sind tatsächlich mächtig, aber nicht die mächtigsten Wesen, deshalb werden wir nicht alle Absichten begreifen können. \footnote{Zumindest nicht, solange wir unser Bewusstsein nicht völlig aufgeweckt haben.}

Um mit einem Model arbeiten zu können, braucht es folgende Regeln:

1) Modelle helfen unseren Gehirnen eine Idee zu verstehen. die Wahl mit einem Modell zu arbeiten kommt aber durch die Beschaffenheit unseres Gehirns zustande und nicht dadurch, dass die Idee, die Information dieses Modell ist.

$\Rightarrow$ Bewusstsein:

Gehirn $\rightarrow$ Modell
$\neq$ Information = Modell

$\Rightarrow$ Modell $\neq$ Wirk-lichkeit, sondern = Abbild 

2) Die Welt passt nicht in EIN Modell!

Deshalb ist es gut eine Vielzahl von Modellen zu kennen und nebeneinander stehen lassen zu können!

Hilfreich: Berkano \textarc{[\withlines]b}: stell Dir die beiden Dreiecke als Modelle vor: Sie können nebeneinander stehen, sind aber sauber getrennt. 

$\Rightarrow$ Vielfalt:

Welt/Universum $\gg$ als $\{1,2,\dots,n \}$ Modelle

$\Rightarrow$ $\sum$ Modellen

$\to$ Berkano \textarc{[\withlines]b} = Mehrere Modelle nebeneinander

$\neq$ vermischt

3) Finde heraus, welches Modell nützlich ist. Finde selbst ein Bild, für das, was Du ausdrücken, darstellen, erklären willst!

Aber, sei konsequent $\Rightarrow$ sei Dir bewusst, wo die Grenzen Deines Modells liegen! 

$\Rightarrow$ Flexibilität: Welches Modell ist für meinen Zweck am Besten geeignet? 

$\Rightarrow$ Für welchen Bereich gilt mein Modell?


$\mathbb{P}$, $\mathbb{N}$ oder $\mathbb{R}$?


4) Wenn Du Dir der Grenzen und des Bereiches Deines Modells sicher bist, dann wirst Du bemerken, wenn Du es entweder

wechseln musst

Bsp. Du hast Primzahlen benutzt $\mathbb{P}$ und bemerkst auch teilbare Zahlen. Dann musst Du das Modell, den Bereich ändern, z.B. in $\mathbb{Z}$


oder Du erweiterst Dein Modell, wenn die Grundlage stimmt, aber Teilbereiche nicht hineinpassen:

Bsp. Du benutzt $\mathbb{N}$ und bemerkst, dass Du die ganzen Zahlen auch benötigst. Dann kannst in diesem Fall Dein Zahlenmodell erweitern, $\mathbb{Z}$\footnote{$\mathbb{N}$ = 1,2,3,4,etc. die natürlichen Zahlen. $\mathbb{R}$ = }
































\section*{Der Körper}

Wichtig! Du bist Dein Körper! Du bist Körper! Was meine ich damit? Es ist nur bedingt sinnvoll den Körper und mich in der Sprache, in Gedanken zu trennen. Dennoch werde ich es in diesem Kapitel hin und wieder machen, denn wir denken heute in diesen Kategorien und ich möchte es nicht kompliziert machen.

Der erste Schritt wird sein Dir Deinen Naturraum wieder spürbar zu machen. Denn dieser ist oft grösser als wir denken. 

Du kann Dich dorthin bewegen. Du hast deine Sinne: Sehen, Hören, Riechen, Schmecken und Tasten zur Verfügung. Alle die Sinneswahrnehmungen landen in Deinem Gehirn. Es sind viele, viele Millionen Daten und Dein Gehirn hat die Aufgabe, die wichtigsten herauszufiltern, die am ehesten zu Deinem Überleben beitragen. 

Dennoch sind alle die vielen Sinneseindrücke da! Weil wir heute so geschützt leben und die meisten mit einer sicheren Höhle (Wohnung) genügend Essen (Geld/Supermarkt/Kühlschrank) versorgt sind, können wir uns viel mehr aussuchen, ob wir unseren Gedanken lauschen oder uns aktiv in unserem Umfeld, der Natur bewegen und daran teilnehmen.

Selbst das Wetter spielt in vielen Breitengraden nur eine Rolle, wenn es zu sehr heftigen Wetterphänomenen kommt. Wenn es z.B. heftig stürmt, sehr trocken ist, etc. Viele sind mit dem Auto unterwegs und können, egal zu welcher Jahreszeit mit Jackett und dünnen Schuhen von ihrem Zuhause mit dem Auto an den Arbeitsplatz gelangen, ohne vom Wetter oder den Jahreszeiten berührt worden zu sein.

Es ist so, wie es ist gut! Denn wir haben, im Gegensatz zu den Tieren und Pflanzen in der Natur und im Gegensatz zu vielen Menschen rund um die Erde genau diese Möglichkeiten. Wir haben diesen Schutz. Wir können und dürfen ihn nutzen und geniessen.


\subsection{Körper und Bewegung}

Alles ist Bewegung. Unser Körper, gemeinhin verstehen wir darunter, alles ausser Kopf und Hals, hier meine ich alles, was physisch sichtbar ist, ist wie stark verlangsamte Bewegung. Damit Du in der Welt physisch sein kannst und in dem physischen Körper sein kannst, kannst Du Dir vorstellen, dass sich alles wie in Slow-Motion bewegt.

Wir haben dadurch Zeit und Raum. Zeit ist ein Raum. Und je langsamer etwas passiert, umso grösser wird der Raum. Wenn Du Dir nun vorstellst wie gross die Erde ist, wie viel Materie es gibt, die diesen Erdenzeitraum füllt, dann kannst Du Dir den Slow-Motion-Effekt vorstellen.

Damit wir fit werden ebenso mit dem geistigen\footnote{was genau mit geistig gemeint ist, dazu später}Teil von uns zu bewusst zu arbeiten, ist es gut unseren Körper als Raum, als inneren Raum kennen zu lernen.

Dazu gibt es einige hilfreiche Werkzeuge:

\subsubsection{Sport}

Ganz einfach! Sport! Oder einfach: Bewegen. Denn je besser Du Deinen Körper, Deine Muskeln, Sehnen, etc, spürst, umso grösser wird ein Teil Deines Inneren Körpergefühls.

Es spielt keine grosse Rolle, welchen Sport Du machst. Mache das, was Dir am meisten Spass macht, oder was Du gerne können möchtest, Bsp. Tanzen, Springen, etc.

Wichtig ist nur, es mit der ganzen Aufmerksamkeit zu tun, wenn Du den Sport machst. Wenn Du Dich bewegst, dann richte Deine Aufmerksamkeit auch auf die Bewegung und Deinen Körper. Begleite ihn! Sei Dein Körper, der sich bewegt.

Und denke nicht gleichzeitig darüber nach, wem Du noch eine Nachricht schicken musst, oder was Du für das Abendessen einkaufen willst.

Du kannst üben, Dich soviel, sooft wie möglich bei den Bewegungen zu spüren, die Du machst: gehen (zur Bushaltestelle), sitzen (Im Bus), kochen, einkaufen, warten, etc.

Du wirst Dich nicht nur besser spüren, sondern Deine Gedanken werden nicht soviel herumquatschen, denn wir können uns nur auf eine Sache gut konzentrieren.

\subsubsection{Sinne}

Das gleiche gilt für die Sinne. Da Du fünf hast, kannst Du hier situativ, tageweise, etc. auswählen, welchem Du bewusst Deine Aufmerksamkeit schenken willst. Du kannst herausfinden, welchen Sinn Du am Liebsten hast und welcher mehr Beachtung bekommen könnte, etc. 

Hier spielt Essen eine Rolle.
Was schmeckt Dir? Wie verändert sich die Lust auf verschiedene Nahrungsmittel, etc.

Da es heute unzählige Ratgeber und Ernährungsformen gibt, werden wir hier nicht weiter über das Thema sprechen.

Wichtig ist für die späteren Gespräche mit den Bäumen, dass Du Deine Sinne kennst und Deine Aufmerksamkeit eine Zeitlang auf ihnen halten kannst.

\subsubsection{Cantienica-Methode}

Wenn Du bei Dir in der Nähe die Möglichkeit hast einen Cantienica-Kurs zu besuchen, dann mache das! 

Ich selbst mache seit 4 Jahren regelmässig einmal die Woche eine Stunde. Ich habe vorher viele Rückenprobleme gehabt. Mit 12 Jahren machte ich schon Krankengymnastik, weil der Arzt einen Verdacht auf Scheuermann hatte. Ich arbeite in der Betreuung von behinderten Kindern und Jugendlichen. Da hatte ich oft Rückenschmerzen. Doch mit der Cantienica-Methode sind diese fast ganz verschwunden.

Aber Cantienica ist mehr als ein Haltungstraining:

Diese Methode trainiert die inneren Muskeln. Damit Du diese findest und trainieren kannst, lernst Du Deine Muskeln von Innen kennen. Die Anleitung der Lehrerin wird Dir helfen, diese zu finden und sie anzuspannen und zu entspannen.

Die Besonderheit für uns Baumspürer liegt darin, dass wir eine weitere Form kennenlernen tief durch das Innere unseres physischen Körpers zu reisen.

Und, Du wirst eine wunderschöne Körperhaltung bekommen. Rückenbeschwerden, Skoliose, Haltungsschäden, Bein-, Hüft- und Fussschmerzen werden gelindert oder verschwinden und Dein innerer Beckenbodenmuskel wird trainiert und das wird Dir in Deiner Sexualität sehr Spass machen! 

Ob das etwas für Männer\dots ? Aah, \dots wegen Beckenboden! Ja, den haben Männer auch\dots ! 

\subsubsection{Craniosacral Therapie}

Höre Dich in Deinem Bekanntenkreis, dem Arzt/der Ärztin Deines Vertrauens um nach einem guten Craniosakraltherapeut.

Diese Methode mit ihren zarten, kaum spürbaren Massagebewegungen ist hocheffizient. Denn sie arbeitet direkt mit der Bewegung der Knochen.

Unsere Knochen bewegen sich! Sie haben eine Art Atembewegung, diese ist winzig und sehr langsam. Unsere Knochen sind der bis auf unsere Zähne, der härtste Teil in unserem Körper. und sie sind wie ein Speicher.

Ein guter Therapeut kann Dir ermöglichen Dich tief in Deinen Körper zu begeben und in Kontakt zu Deinen Knochen zu kommen. 

Ich kann Dir empfehlen auch wenn Dein Rücken gesund ist einmal 1-2 craniosakrale Massagen bei einem guten Therapeuten zu erleben.

Einen guten Therapeut/ eine gute Therapeutin erkennst Du daran, dass Du den Eindruck bekommst er/sie macht nichts und sich Dein Körper beginnt zu bewegen.

\subsection{Gehirn}

Dem Gehirn möchte ich ein eigenes Kapitel geben. Es ist etwas vertrackt unser Verhältnis zu diesem Organ, den das ist es ja. Einerseits geben wir ihm eine riesige Bedeutung. Andererseits nutzen wir nur einen klitzekleinen Teil davon. Im Laufe unseres Lebens beginnt das Gehirn in unserem Kopf zu reden und wir identifizieren uns schliesslich damit. Je nachdem was unser Gehirn uns erzählt und das hängt mit unserer frühkindlichen Prägung und Erfahrung zusammen, fühlen wir uns schliesslich mit uns und unserer Umwelt wohl oder nicht.

selten kommt es uns in den Sinn uns selbst zu fragen, was zum Henker unser Gehirn für einen Unsinn plappern kann den lieben langen Tag!

Mir hat es geholfen mich über das Gehirn zu informieren. mit den verschiedenen Instanzen, die sich dort in meinem Kopf zu Wort melden zum einen, aber auch mit den vielen anderen Aufgaben, die ein Gehirn hat, die nicht mit Worten zu tun haben und die deshalb oft nicht als Arbeit meines Gehirns von mir wahrgenommen werden.

Da uns dieser Körperteil so intensiv in unserem Sein beeinflusst, weil er uns unser Bewusstsein anzeigt und unsere Aufmerksamkeit steuert, wollen wir ihn genauer betrachten.

Wenn unser Gehirn mit uns redet, dann können die Quellen aus unterschiedlichen Bereichen kommen. Das ist wichtig zu wissen! Denn wenn wir uns auf den Weg machen wollen, um  mit den Bäume zu reden, dann ist es gut, wenn wir uns so gut kennen, dass wir zwischen dem, was unser Gehirn sagt und dem was der Baum sagt, unterscheiden können.

Die Wesen in der Natur können sich auf sehr vielfältige Weise äussern und jeder Mensch wird diese Mitteilungen auf seine Weise wahrnehmen. Dies kann ein Gefühl, ein Ton, eine Farbe, ein Geschmack,ein Kribbeln, etc. sein. Dies rauszufinden muss jeder für sich machen.

Aber, je mehr Du über Dich und Deinen Körper weisst, wie er sich anfühlt und wie und wann Dein Gehirn am gesprächigsten wird, umso besser wirst Du spüren, wenn sich der Baum bei Dir meldet. Der Baum kann sich mit Dir nur über Deinen geistigen, Inneren Raum\footnote{Dazu später mehr, eines nach dem anderen!} bemerkbar machen. Wenn Du Dich nicht kennst, dann wirst Du vielleicht seine Stimme wahrnehmen, aber schlicht nicht wissen, dass diese Empfindung, etc. die des Baumes sind!

\subsubsection{Ego}

\tableofcontents

% \printbibliography

\bibliography{Bilbliothek.bib}{}
\bibliographystyle{plain}

\end{document}